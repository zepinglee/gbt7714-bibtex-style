% !TeX encoding = UTF-8
% !TeX program = xelatex
% !TeX spellcheck = en_US

%
% Copyright (c) 2016-2025 Zeping Lee
% Released under the LaTeX Project Public License v1.3c or later versions.
% Repository: https://github.com/zepinglee/gbt7714-bibtex-style
%
\documentclass[10pt,a4paper]{article}

\begin{filecontents}[overwrite]{\jobname.bib}

@standard{gbt7714-2025,
  type = {国家标准},
  title = {信息与文献{$\quad$}参考文献著录规则},
  author = {{全国信息与文献标准化技术委员会}},
  date = {2025-12-02},
  number = {GB/T 7714---2025},
  url = {https://std.samr.gov.cn/gb/search/gbDetailed?id=4507EFE13D37CB6AE06397BE0A0A601F},
  urldate = {2026-01-13},
  langid = {chinese},
}

@software{natbib,
  title = {Natural Sciences Citations and References},
  author = {Daly, Patrick W.},
  date = {2010-09-13},
  url = {https://ctan.org/pkg/natbib},
  urldate = {2026-01-13},
  version = {8.31b},
}

@article{chen2015,
  title = {GB/T 7714新标准对旧标准的主要修改及实施要点提示},
  author = {{陈浩元}},
  date = {2015-08},
  journaltitle = {编辑学报},
  volume = {27},
  number = {4},
  pages = {339--343},
  doi = {10.16811/j.cnki.1001-4314.2015.04.015},
  url = {http://bjxb.cessp.org.cn/ch/reader/view_abstract.aspx?file_no=20150411&flag=1},
  urldate = {2025-03-22},
  langid = {chinese},
}
@article{chen2024dui,
  title = {对GB/T 7714—2015的标识符号问题咨询的答复},
  author = {{陈浩元}},
  date = {2024-05-11},
  journaltitle = {编辑学报},
  volume = {36},
  number = {2},
  pages = {139},
  url = {http://bjxb.cessp.org.cn/ch/reader/view_abstract.aspx?file_no=20240205&flag=1},
  urldate = {2025-03-22},
  langid = {chinese},
}
\end{filecontents}


\usepackage{iftex}
\usepackage{fontspec}
\IfFontExistsTF{SimSun}{\PassOptionsToPackage{fontset=windows}{ctex}}{}

\usepackage[UTF8, scheme=plain, autoindent=true]{ctex}
\usepackage{csquotes}
\usepackage[useregional]{datetime2}

\usepackage{libertineRoman}
\usepackage{biolinum}
\setmonofont{JuliaMono}[
  Scale          = 0.8,
  Extension      = .ttf,
  UprightFont    = *-Regular,
  BoldFont       = *-Bold,
  ItalicFont     = *-RegularItalic,
  BoldItalicFont = *-BoldItalic,
]

\usepackage[verbose,
  top=33mm,
  left=3.5cm, right=3.5cm,
  marginparwidth=2.5cm,
  height=21cm,
  footskip=47.6pt,
  headheight=17pt,
  headsep=20.40001pt,
]{geometry}
\setlength\parindent{1em}
\usepackage{microtype}
\usepackage{ragged2e}
\usepackage{hyphenat}
\usepackage{booktabs}
\usepackage{multicol}
\usepackage[svgnames]{xcolor}
\usepackage{longtable}
\usepackage{array}
\newcolumntype{L}[1]{%
  >{\raggedright\let\newline\\\arraybackslash\hspace{0pt}}p{#1}}
\usepackage[listings, breakable, skins]{tcolorbox}%
\usetikzlibrary{arrows.meta}
\usetikzlibrary{shapes.geometric}
\usetikzlibrary{fit}
\usetikzlibrary{positioning}

% \usepackage[style=ext-authoryear-iecomp, backend=biber, labelnumber]{biblatex}
% \addbibresource{biblatex-examples.bib}
% \addbibresource{biblatex-ext-examples.bib}
% \usepackage{biblatex-ext-tabular}
% \usepackage{biblatex-ext-oa}
% % Normally, the following packages should not be loaded explicitly.
% % Instead, one of them (and *only* one) should be loaded via
% % biblatex-ext-oa's options.
% \usepackage{biblatex-ext-oasymb-tikz}
% % We need to undefine \oasymbol to be able to load the other packages as well.
% \undef\oasymbol
% \undef\DefineOASymbol
% \usepackage{biblatex-ext-oasymb-l3draw}
% % We need to undefine \oasymbol to be able to load the other packages as well.
% \undef\oasymbol
% \undef\DefineOASymbol
% \usepackage{biblatex-ext-oasymb-pict2e}

% \DeclareFieldFormat{bibentrysetcount}{\mkbibparens{\mknumalph{#1}}}
% \DeclareFieldFormat{labelnumberwidth}{\mkbibbrackets{#1}}
% \DeclareFieldFormat{shorthandwidth}{\mkbibbrackets{#1}}

% \defbibenvironment{bibliographyNUM}
%   {\list
%      {\printtext[labelnumberwidth]{%
%         \printfield{labelprefix}%
%         \printfield{labelnumber}}}
%      {\setlength{\labelwidth}{\labelnumberwidth}%
%       \setlength{\leftmargin}{\labelwidth}%
%       \setlength{\labelsep}{\biblabelsep}%
%       \addtolength{\leftmargin}{\labelsep}%
%       \setlength{\itemsep}{\bibitemsep}%
%       \setlength{\parsep}{\bibparsep}}%
%       \renewcommand*{\makelabel}[1]{\hss##1}}
%   {\endlist}
%   {\item}

% \newcounter{extblxdoc@examplebib}

\makeatletter
% \defbibcheck{examplebib}{%
%   \xifinlist{\thefield{entrykey}}{\extblxdoc@examplebib@list}
%     {}
%     {\skipentry}}

% \newcommand*{\extblxdoc@fixexamplelinks}{%
%   \protected\def\blx@anchor{%
%     \xifinlist{\the\c@refsection @\the\c@extblxdoc@examplebib
%                @\abx@field@entrykey}{\blx@anchors}
%       {}
%       {\listxadd\blx@anchors{\the\c@refsection @\the\c@extblxdoc@examplebib
%                              @\abx@field@entrykey}%
%        \hyper@natanchorstart{\the\c@refsection @\the\c@extblxdoc@examplebib
%                              @\abx@field@entrykey}%
%        \hyper@natanchorend}}%
%   \long\def\blx@bibhyperref[##1]##2{%
%      \blx@sfsave\hyper@natlinkstart{\the\c@refsection
%        @\the\c@extblxdoc@examplebib @##1}\blx@sfrest
%      ##2%
%      \blx@sfsave\hyper@natlinkend\blx@sfrest}%
% }

\usepackage{gbt7714}
\bibliographystyle{gbt7714-2025-numeric}

\let\bibfield\relax

\newcommand*{\exampleprintbib}[2][]{%
  \nocite{#2}%
  \let\extblxdoc@examplebib@list\empty
  \def\do##1{\listeadd\extblxdoc@examplebib@list{\detokenize{##1}}}%
  \docsvlist{#2}%
  \AtNextBibliography{%
    \stepcounter{extblxdoc@examplebib}%
    \extblxdoc@fixexamplelinks}%
  \printbibliography[check=examplebib, heading=none, #1]}

\newcommand*{\exampleprintbibtab}[1]{%
  \nocite{#1}%
  \let\extblxdoc@examplebib@list\empty
  \def\do##1{\listeadd\extblxdoc@examplebib@list{\detokenize{##1}}}%
  \docsvlist{#1}%
  \AtNextBibliography{%
    \stepcounter{extblxdoc@examplebib}%
    \extblxdoc@fixexamplelinks}%
  \printbibtabular[check=examplebib, heading=none]}


\usepackage{ltxdockit}
\usepackage{btxdockit}
\usepackage{cleveref}
\hypersetup{%
  colorlinks=true,
  allcolors=spot,
  bookmarksopen=false,
  bookmarksnumbered=false,
  plainpages=false}

\definecolor{highlight1}{RGB}{0,153, 153}
\definecolor{highlight2}{RGB}{240, 0, 0}
\definecolor{spot}{rgb}{0,0.2,0.6}

\lstdefinestyle{extblxstylegeneral}{%
  aboveskip    = {0\p@ \@plus 6\p@},
  belowskip    = {0\p@ \@plus 6\p@},
  tabsize      = 2,
  breaklines   = true,
  breakatwhitespace = true,
  keepspaces   = true,
  escapeinside = {(*@}{@*)},
  moredelim    = {[is][\ttfamily\bfseries\color{highlight1}]{|}{|}},
  moredelim    = {[is][\ttfamily\bfseries\color{highlight1}]{|1}{1|}},
  moredelim    = {[is][\ttfamily\bfseries\color{highlight2}]{|2}{2|}},
}

\lstdefinelanguage{extBibTeX}{%
  morekeywords=[1]{%
    @article,@artwork,@audio,@bibnote,@book,@bookinbook,@booklet,%
    @collection,@commentary,@customa,@customb,@customc,@customd,%
    @custome,@customf,@inbook,@incollection,@inproceedings,%
    @inreference,@image,@jurisdiction,@legal,@legislation,@letter,%
    @dataset,
    @manual,@misc,@movie,@music,@mvcollection,@mvreference,%
    @mvproceedings,@mvbook,@online,@patent,@performance,@periodical,%
    @proceedings,@reference,@report,@review,@set,@software,@standard,%
    @suppbook,@suppcollection,@suppperiodical,@thesis,@unpublished,@video%
   },
   morekeywords=[2]{author,title,date,journal,volume,number,pages,doi,
                    eprint,eprinttype,eprintclass},
   keywordstyle=[1]{\bfseries\spotcolor},
   keywordstyle=[2]{\spotcolor},
   sensitive=false,
}

\lstdefinestyle{extblxstylelatex}{%
  language  = {[LaTeX]TeX},
  style     = {extblxstylegeneral},
  moretexcs = {
    dimexpr,arraybackslash,newcolumntype,anchor,driver,plain,plainlang,
    anchorlang,
    thefield,bibstring,
    addbibresource,
    setlength,bibhang,addcomma,adddot,addperiod,addcolon,addspace,
    addnbspace,
    mkbibbold,mkbibemph,mkbibbrackets,mkbibparens,
    usebibmacro,newbibmacro,renewbibmacro,setunit,newunit,printfield,printlist,
    bibopenparen,bibcloseparen,bibopenbracket,bibclosebracket,
    iflistundef,iffieldundef,ifnameundef,iffieldnums,
    ExecuteBibliographyOptions,
    defbibtabular,defbibtabulartwocolumn,defbibenvironment,
    printbibliography,printbibtabular,
    autocite,
    DeclareFieldFormat,DeclareDelimFormat,
    DeclareDelimcontextAlias,UndeclareDelimcontextAlias,
    DeclareInnerCiteDelims,UndeclareInnerCiteDelims,DeclareInnerCiteDelimsAlias,
    DeclareOuterCiteDelims,UndeclareOuterCiteDelims,DeclareOuterCiteDelimsAlias,
    UndeclareCiteDelims,
    introcitepunct,volnumdelim,maintitletitledelim,voltitledelim,sernumdelim,
    jourvoldelim,
    volnumdatedelim,locdatedelim,locpubdelim,publocdelim,pubdatedelim,
    extradateonlycompcitedelim,introcitesep,introcitewidth,introcitebreak,
    DeclareOpenAccessFieldUrl,DeclareOpenAccessEprintUrl,
    DeclareOpenAccessEprintAlias,DeclareOpenAccessUrlFieldPriority},
}

\lstdefinestyle{extblxstylebibtex}{%
  language  = {extBibTeX},
  style     = {extblxstylegeneral},
}


\newcommand*{\highlight}[2][1]{\textcolor{highlight#1}{#2}}
\newcommand*{\highlightbf}[2][1]{\textcolor{highlight#1}{\textbf{#2}}}
% \DeclareFieldFormat{highlight1}{\textcolor{highlight1}{#1}}
% \DeclareFieldFormat{highlight2}{\textcolor{highlight2}{#1}}

\newtcolorbox{bibexamplecolorbox}[1][]{enhanced,
                               colframe=spot!75!black, colback=spot!5!white,
                               before title=\raggedright,
                               #1}

\newenvironment{biblist}{%
  %  \bibsection
  \parindent\z@
  \bibpreamble
  \bibfont
  \list{\@biblabel{\the\c@NAT@ctr}}{\@bibsetup{9}\global\c@NAT@ctr\z@}%
  \ifNAT@openbib
    \renewcommand\newblock{\par}%
  \else
    \renewcommand\newblock{\hskip .11em \@plus.33em \@minus.07em}%
  \fi
  \sloppy\clubpenalty4000\widowpenalty4000
  \sfcode`\.\@m
  \let\NAT@bibitem@first@sw\@firstoftwo
     \let\citeN\cite \let\shortcite\cite
     \let\citeasnoun\cite
}{%
  \bibitem@fin
  \bibpostamble
  \def\@noitemerr{%
   \PackageWarning{natbib}{Empty `thebibliography' environment}%
  }%
  \endlist
  \bibcleanup
}

\newenvironment{citeexample}[1][]{%
  \begin{bibexamplecolorbox}[#1]%
}{%
  \end{bibexamplecolorbox}
}

\newenvironment{bibexample}[1][]{%
  \begin{bibexamplecolorbox}[#1]%
    \setcitestyle{numbers}%
    \begin{biblist}%
}{%
    \end{biblist}
  \end{bibexamplecolorbox}
}

\newenvironment{numericbibexample}[1][]{%
  \begin{bibexamplecolorbox}[#1]%
    \setcitestyle{numbers}%
    \begin{biblist}%
}{%
    \end{biblist}
  \end{bibexamplecolorbox}
}

\newenvironment{authoryearbibexample}[1][]{%
  \begin{bibexamplecolorbox}[#1]%
    \setcitestyle{authoryear}%
    \begin{biblist}%
}{%
    \end{biblist}
  \end{bibexamplecolorbox}
}

\newtcblisting{biblatexcode}{enhanced,
                             colframe=black!75!white, colback=black!5!white,
                             listing only,
                             frame hidden,
                             breakable,
                             listing style=extblxstylelatex}
\newtcblisting{latexcode}{enhanced,
                             colframe=black!75!white, colback=black!5!white,
                             listing only,
                             frame hidden,
                             breakable,
                             listing style=extblxstylelatex}
\newtcblisting{bibtexfile}{enhanced,
                           colframe=black!75!white, colback=black!5!white,
                           listing only,
                           frame hidden,
                           breakable,
                           listing style = extblxstylebibtex}

\newtcbinputlisting{\inputexamplebibfile}[2][]{%
  listing file={#2},
  enhanced,
  colframe=black!75!white, colback=black!5!white,
  listing only,
  frame hidden,
  breakable,
  listing style = extblxstylebibtex,
  #1}

\newtcolorbox{warnbox}[1][]{%
  enhanced,
  before upper={
    \parskip=0.5\baselineskip
    \advance\parskip by 0pt plus 2pt
  },
  colframe=red, colback=red!5!white,
  underlay={%
    \path[draw=none] (interior.south east)
                     rectangle node {\warnsymbol}
                     ([xshift=15mm,yshift=.1cm]interior.north east);},
  #1}

\newtcolorbox{remindbox}[1][]{%
  enhanced,
  before upper={
    \parskip=0.5\baselineskip
    \advance\parskip by 0pt plus 2pt
  },
  colframe=yellow, colback=yellow!5!white,
  underlay={%
    \path[draw=none] (interior.south east)
                     rectangle node {\dbendsymbol}
                     ([xshift=15mm,yshift=.1cm]interior.north east);},
  #1}

\tikzset{
  MWEcommentbox/.style={font=\sffamily\footnotesize, gray!60!darkgray},
}

% \AtUsedriver*{%
%   \delimcontext{bib}%
%   \let\newblock\relax
%   \let\abx@macro@bibindex\@empty
%   \let\abx@macro@pageref\@empty}


% this is taken from ltxdockit.cls, which is not loaded
\newrobustcmd*{\fnurl}[1][]{\hyper@normalise\ltd@fnurl{#1}}
\def\ltd@fnurl#1#2{\footnote{#1\hyper@linkurl{\Hurl{#2}}{#2}}}

\newrobustcmd*{\email}{\hyper@normalise\ltd@email}
\def\ltd@email#1{\href{mailto:#1}{#1}}

% title
\providecommand*{\titlepage}{}
\providecommand*{\titlefont}{}
\renewrobustcmd*{\titlepage}[1]{\setkeys{ltd@ttp}{#1}}
\renewcommand*{\titlefont}{\bfseries}
\define@key{ltd@ttp}{title}{\def\ltd@title@title{#1}}
\define@key{ltd@ttp}{subtitle}{\def\ltd@title@subtitle{#1}}
\define@key{ltd@ttp}{url}{\def\ltd@title@url{#1}}
\define@key{ltd@ttp}{author}{\def\ltd@title@author{#1}}
\define@key{ltd@ttp}{email}{\def\ltd@title@email{#1}}
\define@key{ltd@ttp}{revision}{\def\ltd@title@revision{#1}}
\define@key{ltd@ttp}{date}{\def\ltd@title@date{#1}}

\providecommand*{\printtitlepage}{}
\renewrobustcmd*{\printtitlepage}{%
  \begingroup
  \centering\titlefont
  \begingroup\LARGE
    \ifundef\ltd@title@url
      {\ltd@title@title}
      {\href{\ltd@title@url}{\ltd@title@title}}%
    \par
  \endgroup
  \vspace{0.25\baselineskip}
  \begingroup\large
    \ltd@title@subtitle\par
  \endgroup
  \expandafter\url\expandafter{\ltd@title@url}\par
  \begin{multicols}{2}
  \raggedleft
    \ltd@title@author\par
    \expandafter\email\expandafter{\ltd@title@email}\par
  \raggedright
    Version \ltd@title@revision\par\ltd@title@date
  \end{multicols}
  \endgroup}


\renewcommand\tableofcontents{%
    \pdfbookmark[1]{\contentsname}{contents}%
    \section*{\contentsname
        \@mkboth{%
           \MakeUppercase\contentsname}{\MakeUppercase\contentsname}}%
    \@starttoc{toc}%
    }
\def\@starttoc#1{%
  \begingroup
    \begin{multicols}{2}
    \makeatletter
    \RaggedRight
    \@input{\jobname.#1}%
    \if@filesw
      \expandafter\newwrite\csname tf@#1\endcsname
      \immediate\openout \csname tf@#1\endcsname \jobname.#1\relax
    \fi
    \end{multicols}
    \@nobreakfalse
  \endgroup}

\newrobustcmd*{\tex}{\TeX}
\newrobustcmd*{\etex}{\mbox{e-TeX}}
\newrobustcmd*{\pdftex}{pdf\-\tex}
\newrobustcmd*{\xetex}{Xe\-\tex}
\newrobustcmd*{\luatex}{Lua\-\tex}
% \newrobustcmd*{\latex}{\LaTeX}%{La\kern-0.07em TeX}
\newrobustcmd*{\pdflatex}{pdf\-\latex}
\newrobustcmd*{\xelatex}{Xe\-\latex}
\newrobustcmd*{\lualatex}{Lua\-\latex}
\newrobustcmd*{\miktex}{Mik\-\tex}
\newrobustcmd*{\texlive}{\tex~live}
\newrobustcmd*{\bibtex}{Bib\kern-0.07em TeX}
\newrobustcmd*{\lppl}{\latex{} Project Public License}
\newrobustcmd*{\pdf}{\acr{PDF}}
\newrobustcmd*{\utf}{\mbox{\acr{UTF}-8}}
\newrobustcmd*{\pgftikz}{PGF/Ti\emph{k}Z}

\pdfstringdefDisableCommands{%
  \def\tex{TeX}%
  \def\etex{e-TeX}%
  \def\xetex{XeTeX}%
  \def\latex{LaTeX}%
  \def\xelatex{XeLaTeX}%
  \def\bibtex{BibTeX}%
  \def\lppl{LaTeX Project Public License}%
  \def\pdf{PDF}%
  \def\utf{UTF-8}%
}

\let\accentcolour\spotcolor

\newcommand*{\allsectionsfont}{\sffamily\accentcolour}

\renewcommand\section{\@startsection {section}{1}{\z@}%
                                     {-3.5ex \@plus -1ex \@minus -.2ex}%
                                     {2.3ex \@plus.2ex}%
                                     {\normalfont\Large\bfseries\allsectionsfont}}
\renewcommand\subsection{\@startsection{subsection}{2}{\z@}%
                                       {-3.25ex\@plus -1ex \@minus -.2ex}%
                                       {1.5ex \@plus .2ex}%
                                       {\normalfont\large\bfseries\allsectionsfont}}
\renewcommand\subsubsection{\@startsection{subsubsection}{3}{\z@}%
                                       {-3.25ex\@plus -1ex \@minus -.2ex}%
                                       {1.5ex \@plus .2ex}%
                                       {\normalfont\normalsize\bfseries\allsectionsfont}}
\renewcommand\paragraph{\@startsection{paragraph}{4}{\z@}%
                                      {3.25ex \@plus1ex \@minus.2ex}%
                                      {-1em}%
                                      {\normalfont\normalsize\bfseries\allsectionsfont}}
\renewcommand\subparagraph{\@startsection{subparagraph}{5}{\parindent}%
                                         {3.25ex \@plus1ex \@minus .2ex}%
                                         {-1em}%
                                         {\normalfont\normalsize\bfseries\allsectionsfont}}

% from KOMA-Script
\newcommand*{\@list@extra}{%
  \ifdim\parskip>\z@
    \topsep\z@
    \parsep\parskip
    \itemsep\z@
  \fi
}
\appto\@listi{\@list@extra}
\appto\@listii{\@list@extra}
\appto\@listiii{\@list@extra}
\appto\@listiv{\@list@extra}
\appto\@listv{\@list@extra}
\appto\@listvi{\@list@extra}

\renewcommand*{\verbatimfont}{\ttfamily}
\renewcommand*{\displayverbfont}{\ttfamily}
\renewcommand*{\marglistfont}{\accentcolour\sffamily\small}
\renewcommand*{\margnotefont}{\sffamily\small}
\renewcommand*{\optionlistfont}{\accentcolour\sffamily\displayverbfont}
\renewcommand*{\ltxsyntaxfont}{\ttfamily}
\renewcommand*{\ltxsyntaxlabelfont}{\accentcolour\displayverbfont}
\renewcommand*{\changelogfont}{\normalfont}
\renewcommand*{\changeloglabelfont}{\accentcolour\sffamily\bfseries}
\newcommand*{\stylelistlabelfont}{\accentcolour\sffamily\small}
\newcommand*{\bibfieldformatfont}{\sffamily}
\newcommand*{\bibfieldformatlabelfont}{\accentcolour\bibfieldformatfont\small}

\newenvironment*{stylelist}
  {\list{}{%
     \setlength{\labelwidth}{\marglistwidth}%
     \setlength{\labelsep}{\marglistsep}%
     \setlength{\leftmargin}{0pt}%
     \renewcommand*{\makelabel}[1]{\hss\stylelistlabelfont##1}}%
   \def\styleitem##1{%
     \item[{##1}]%
     \ltd@pdfbookmark{##1}{##1}}}
  {\endlist}

\newenvironment*{bibfieldformatlist}
  {\list{}{%
     \setlength{\labelwidth}{\marglistwidth}%
     \setlength{\labelsep}{\marglistsep}%
     \setlength{\leftmargin}{0pt}%
     \renewcommand*{\makelabel}[1]{\hss\bibfieldformatlabelfont##1}}%
   \def\bibfieldformatitem##1{%
     \item[{##1}]%
     \ltd@pdfbookmark{##1}{##1}}}
  {\endlist}

\newenvironment*{keymarglistbook}
  {\marglist
   \setlength{\itemsep}{0pt}%
   \raggedright
   \let\@@item\item
   \def\keyitem[##1]{%
     \@@item[{##1}]%
     \ltd@pdfbookmark{##1}{##1}}}
  {\endmarglist}

% modified for correct measurements
\def\ltd@option@i#1#2#3#4[#5]{%
  \item[#2]%
  \ltd@pdfbookmark{#1}{#1}%
  \begingroup\raggedright
  \ltd@textverb{=}%
  \settowidth\@tempdimb{\ltd@textverb{=}}%
  \settowidth\@tempdimc{\optionlistfont#2}%
  \ifdim\@tempdimc>\marglistwidth
    \@tempdimc=\dimexpr\@tempdimc-\marglistwidth\relax
  \else
    \@tempdimc=0pt
  \fi
  \@tempdima=\dimexpr\linewidth-\@tempdimb-\@tempdimc\relax
  \ifblank{#4}
    {}
    {\settowidth\@tempdimb{默认:#4}%
     \@tempdima=\dimexpr\@tempdima-\@tempdimb-2em\relax}%
  \parbox[t]{\@tempdima}{\raggedright #3}%
  \ifblank{#4}
    {}
    {\hfill 默认:#4}%
  \ifblank{#5}
    {}
    {\marginpar{\margnotefont #5}}%
  \par\endgroup
  \nobreak\vspace{\itemsep}}

% inject a label additional to the bookmark so we can link stuff
\let\exltd@pdfbookmark\ltd@pdfbookmark
\def\ltd@pdfbookmark#1#2{%
  \phantomsection\label{exltd@itm@#1}%
  \exltd@pdfbookmark{#1}{#2}}

% \blx@inputonce{ext-biblatex-aux.def}{auxiliary code for ext-biblatex}
%   {}{}{}{}
\newcommand*{\biblatexversion}{\extblx@requiredbiblatexversion}
\def\exltd@isofydate#1/#2/#3{#1-#2-#3}
\newcommand*{\biblatexdate}{\extblx@requiredbiblatexdate}
% \expandafter\def\expandafter\biblatexdate\expandafter{%
%   \expandafter\exltd@isofydate\extblx@requiredbiblatexdate}

\AtEndPreamble{%
  \deflength{\marglistwidth}{(\oddsidemargin+2cm)*85/100}}

\newcommand{\tikzmark}[2][]{\tikz[overlay,remember picture] \node (#2) {#1};}

\newcommand*{\tikztextmark}[2]{%
  \tikz[remember picture,baseline,inner sep=0pt]\node [anchor=base] (#1) {#2};}

% *{<cmd>}{<x-shift>}{<y-shift>}
% unstarred version for commands defined by biblatex-ext
% starred version for standard biblatex commands
\def\punctarrow{%
  \@ifstar\punctarrow@ii\punctarrow@i}

\def\punctarrow@i{%
  \def\punctarrow@cmdfont{\bfseries}%
  \def\punctarrow@hyperref##1##2{%
    \hyperref[exltd@itm@##1]{##2}%
  }%
  \punctarrow@iii
}

\def\punctarrow@ii{
  \def\punctarrow@cmdfont{\itshape}%
  \let\punctarrow@hyperref\@secondoftwo
  \punctarrow@iii
}

\def\punctarrow@iii#1#2#3{%
  \ifdimcomp{#2}{<}{0pt}
    {\def\punctarrow@nodeanchor{east}}
    {\def\punctarrow@nodeanchor{west}}%
  \draw[spot,thick,latex-,rounded corners] (#1) |- ++ (#2,#3)
    node[anchor=\punctarrow@nodeanchor,text = black]
    {\punctarrow@hyperref{#1}{\punctarrow@cmdfont\cmd{#1}}};%
}

% By Stefan Kottwitz, see https://tex.stackexchange.com/a/799/35864
\newcommand*\ttjustify{%
  \fontdimen2\font=0.4em% interword space
  \fontdimen3\font=0.2em% interword stretch
  \fontdimen4\font=0.1em% interword shrink
  \fontdimen7\font=0.1em% extra space
  \hyphenchar\font=`\-% allowing hyphenation
}

% no \mbox here, we may have to break things
\renewrobustcmd*{\sty}[1]{{\verbatimfont\ttjustify #1}}
\newrobustcmd*{\blxstyle}[1]{{\verbatimfont\ttjustify #1}}
\newrobustcmd*{\filename}[1]{{\verbatimfont\ttjustify #1}}
% \renewrobustcmd*{\bibfield}[1]{\sty{#1}}
\newrobustcmd*{\pkg}[1]{\sty{#1}}
\renewrobustcmd*{\opt}[1]{\sty{#1}}
\newrobustcmd*{\optval}[1]{\sty{#1}}
\newrobustcmd*{\bibmacro}[1]{\sty{#1}}
\renewrobustcmd*{\bibtype}[1]{\sty{@#1}}
\renewrobustcmd*{\cmd}[1]{\sty{\textbackslash #1}}
\let\cs\cmd
\newrobustcmd*{\bibfieldformat}[1]{{\bibfieldformatfont #1}}

\newrobustcmd*{\gencode}[1]{{\verbatimfont\ttjustify #1}}

\def\exltd@buildhypercmd@i#1{%
  \expandafter\newrobustcmd\expandafter*\expandafter
    {\csname hyper#1@i\endcsname}[2][]{%
    \hyperref[exltd@itm@##1]{\csname #1\endcsname{##2}}}
  \csdef{hyper#1}{\@dblarg{\csname hyper#1@i\endcsname}}}

\def\exltd@buildhypercmd{\forcsvlist{\exltd@buildhypercmd@i}}

\exltd@buildhypercmd{cmd,opt,bibmacro,len,blxstyle,sty,bibfield,bibfieldformat,
  filename}

\newrobustcmd*{\hyperkvopt}[2]{{%
  \verbatimfont\hyperref[exltd@itm@#1]{#1}\penalty\@M
  \hskip 0em plus 0.15em\relax
  =\penalty\hyphenpenalty
  \hskip 0em plus 0.15em\relax #2}}

\newcommand*{\ctan}{\mkbibacro{CTAN}}
\newcommand*{\gitbaseurl}{https://github.com/zepinglee/gbt7714-bibtex-style}
\newcommand*{\extblxversion}{3.0.0-beta}
\newcommand*{\biber}{Biber}
\newcommand*{\gitissuelink}[1]{%
  \href{\gitbaseurl/issues/#1}{issue \##1 on github}}

\newrobustcmd*{\CSdelim}{%
  \textcolor{spot}{\margnotefont\footnotesize context sensitive}}
\newrobustcmd*{\CSdelimMark}{%
  \leavevmode\marginpar{\CSdelim}}

\newcommand*{\mpdl}{$\langle$}
\newcommand*{\mpdr}{$\rangle$}

\iftutex
  \newcommand*{\hmpdl}{$\langle$}
  \newcommand*{\hmpdr}{$\rangle$}
\else
  \newcommand*{\hmpdl}{$\bm{\langle}$}
  \newcommand*{\hmpdr}{$\bm{\rangle}$}
\fi

\def\textvisiblespace{%
  \raisebox{-2.2pt}{%
    \mbox{\kern.04em\vrule \@height.5ex \@width.12ex}%
    \vbox{\hrule \@width.2em \@height.12ex}%
    \hbox{\vrule \@height.5ex \@width.12ex}%
    \kern.04em}}

% Bourbaki dangerous bend symbol by Heiko Oberdiek
% https://tex.stackexchange.com/users/16967/heiko-oberdiek
% https://tex.stackexchange.com/a/262510/35864
\newcommand*{\dbendsymbol@tikz}{%
  \begin{tikzpicture}[
    line cap=but,
    line join=round,
    x=1.2em,
    line width=2pt,
    y=2*(height("Z")-\pgflinewidth)*(1-sin(10)),
    rotate=-10,
    rounded corners=1.5pt,]
    \draw (.5,.5) node[scale=2,draw,diamond,fill=yellow,color=yellow] {};
    \draw (1, 0) -- (0, 0) -- (1, 1) -- (0, 1);
  \end{tikzpicture}}

\newsavebox{\dbendsymbol@box}
\sbox{\dbendsymbol@box}{\dbendsymbol@tikz}
\newcommand*{\dbendsymbol}{\usebox{\dbendsymbol@box}}

\newcommand*{\warnsymbol@tikz}{%
  \begin{tikzpicture}[
    line cap=but,
    line join=round,
    line width=2pt,
    rounded corners=1.5pt,]
    \node[regular polygon, regular polygon sides=3,
          scale=1.2, inner sep=-.22em,
           draw=red]
      {\raisebox{.26em}{\LARGE\bfseries !}};
  \end{tikzpicture}}


\usepackage{threeparttable}

\newsavebox{\warnsymbol@box}
\sbox{\warnsymbol@box}{\warnsymbol@tikz}
\newcommand*{\warnsymbol}{\usebox{\warnsymbol@box}}

\makeatother

\titlepage{%
  title    = {GB/T 7714 \hologo{BibTeX} style},
  subtitle = {},
  url      = {\gitbaseurl},
  author   = {Zeping Lee},
  email    = {zepinglee AT gmail DOT com},
  revision = {\extblxversion},
  date     = {\DTMDate{2025-12-27}},
}

\hypersetup{%
  pdftitle    = {GB/T 7714 BibTeX style},
  pdfsubject  = {BibTeX styles for GB/T 7714},
  pdfauthor   = {Zeping Lee},
  pdfkeywords = {bibtex, GB/T 7714, bst, citation style, bibliography},
}

\hyphenation{%
  star-red
  un-star-red
  bib-lio-gra-phy
  white-space
  bib-open-paren
  bib-close-paren
  bib-open-bracket
  bib-close-bracket
  main-title-after-title
  jour-vol-delim
  in-name-before-title
  tikz-picture
  Define-OA-Symbol
}

\usepackage{doc}
\usepackage{hologo}

\linespread{1.3}

\MakeShortVerb{\|}

\newcommand\entrytype[1]{\texttt{@#1}}
\newcommand\entrykey[1]{\texttt{#1}}
\newcommand\field[1]{\texttt{#1}}
% \newcommand\BibTeX{BibTeX}

\def\doi#1{\href{https://doi.org/#1}{\nolinkurl{#1}}}
\def\cstr#1{\href{https://cstr.cn/#1}{\nolinkurl{#1}}}


\begin{document}

% \maketitle
\printtitlepage
\tableofcontents

\begin{abstract}
The \pkg{gbt7714} package provides a \BibTeX{} implementation for the China's
national bibliography style standard GB/T 7714.
It consists of \file{.bst} files for numeric and author-date styles as well as a
\LaTeX{} package which provides the citation style defined in the standard.
It is compatible with \pkg{natbib} and supports language detection (Chinese
and English) for each biblilography entry.
\end{abstract}


\section{简介}
\label{sec:intro}

GB/T 7714—2025 《信息与文献\quad 参考文献著录规则》\cite{gbt7714-2025}
(以下简称“国标”)是中国的参考文献格式推荐性标准,
广泛应用于国内的学术期刊、学位论文等。
对于 \LaTeX{} 用户,\BibTeX{} 是常用的自动化参考文献生成工具。
本项目旨在实现 GB/T 7714 的 \BibTeX{} 样式,
使作者在进行学术写作时,更方便地生成符合国标要求的参考文献表和引用标注。

本宏包的主要特性有:
\begin{itemize}
  \item 支持“顺序编码制”(numeric)和“著者-出版年制”(author-year)两种标引体系。
  \item 自动识别中、英文语言并进行相应处理。
  \item 提供接口供用户修改格式细节。
  \item 兼容 \pkg{biblatex} 风格的 \file{.bib} 数据库。
  \item 同时提供了 2005、2015 版的 \file{.bst} 文件。
\end{itemize}

本宏包的主页:\url{https://github.com/zepinglee/gbt7714-bibtex-style}。

文档中假定用户已经具备一定的 \LaTeX{} 和 \BibTeX{} 使用经验。
如果作者刚开始接触 \BibTeX{},建议先阅读 lshort-zh-cn 和 BibTeX 的文档。


\section{安装}

本宏包已经在常见的 \TeX{} 发行版中附带,
用户可以直接在文档中调用。

如果需要使用最新版本,而发行版的包管理工具无法升级,则需要手动安装 \pkg{gbt7714}。
用户需要从本项目的
\href{https://github.com/zepinglee/gbt7714-bibtex-style}{GitHub 主页}
下载 \file{gbt7714.sty} 和需要的 \file{.bst} 文件,
并将它们与主文档放在同一目录下。
\file{gbt7714.sty} 依赖 \pkg{natbib} 宏包,并按照国标的规定修改正文中的引用格式。
\file{.bst} 控制 \BibTeX{} 生成的参考文献表格式。

如果用户需要修改 `.bst` 文件的名称,可以参考第 \ref{sec:customize} 节的内容。


\section{使用说明}
\label{sec:usage}

下面是 \pkg{gbt7714} 宏包的一个简单示例。

\begin{latexcode}
  \documentclass[UTF8]{ctexart}
  \usepackage[sort]{gbt7714}
  \bibliographystyle{gbt7714-numeric}
  \begin{document}
  \cite{citekey}
  \bibliography{bibfile}
  \end{document}
\end{latexcode}

用户需要在导言区调用 \pkg{gbt7714} 宏包,在可选参数中可以设置选项。
这些选项也可以使用 \cs{gbtbstsetup} 命令进行设置(见第 \ref{sec:option} 节)。

\cs{bibliographystyle} 命令可以选择参考文献表的样式文件,
\pkg{gbt7714} 提供的全部样式见第 \ref{sec:styles} 节。

正文中,用户可以使用 \cs{cite} 命令进行引用,
其中 \entrykey{citekey} 是文献的标识符(key),
对应 \file{.bib} 数据库中的文献条目。
此外,\pkg{gbt7714} 依赖的 \pkg{natbib} 宏包还提供了其他引用命令,
详见第 \ref{sec:cite} 节。

文末的参考文献表可以通过 \cs{bibliography} 命令生成。
示例中调用了 \file{bibfile.bib} 的 BibTeX 数据库。
\file{.bib} 文件的格式及本宏包的填写要求见第 \ref{sec:bibfile} 节。




\section{参考文献样式}

按照国标的规定,参考文献的标注体系分为“顺序编码制”和
“著者-出版年制”。
用户应在导言区调用宏包 \pkg{gbt7714},并且使用 \cs{bibliographystyle}
命令选择参考文献表的样式,例如:
\begin{latexcode}
  \bibliographystyle{gbt7714-numeric}  % 顺序编码制
\end{latexcode}
或者
\begin{latexcode}
  \bibliographystyle{gbt7714-authoryear}  % 著者-出版年制
\end{latexcode}

不带年份的样式名称是现行的 GB/T 7714 版本。
用户也可以指定带年份的版本,如 \texttt{gbt7714-2025-numeric}。
下面列举了可用的参考文献表样式文件:

\begin{itemize}
  \item \file{gbt7714-numeric}
  \item \file{gbt7714-authoryear}
  \item \file{gbt7714-2025-numeric}
  \item \file{gbt7714-2025-authoryear}
  \item \file{gbt7714-2015-numeric}
  \item \file{gbt7714-2015-authoryear}
  \item \file{gbt7714-2005-numeric}
  \item \file{gbt7714-2005-authoryear}
\end{itemize}

注意,\pkg{gbt7714} v3.0.0 修改了样式的后缀名称,
将 \file{-numerical} 和 \file{-author-year} 分别改为
\file{-numeric} 和 \file{-authoryear}。
这样与其他 Bib(La)TeX 的样式名称更一致。
旧版的样式名称仍然可以使用,但是会警告用户使用新名称。


\section{正文中的引用命令}
\label{sec:cite}

在正文中进行引用时。
\LaTeX{} 提供了最基础的命令 \cs{cite},
但是这还不能满足国标规定的两种参考文献标引体系的各种格式。
\pkg{gbt7714} 在内部调用了 \pkg{natbib} 宏包来处理正文中的引用标注,
因此用户可以使用 \pkg{natbib} 提供的各种引用命令,
详见该宏包的文档\cite{natbib}。

\subsection{顺序编码制}

顺序编码制是按照文献被引的先后顺序、以序号标注的标引体系。
国标要求将序号置于方括号“[ ]”中,在示例中均使用上标的格式著录。
用户在调用 \pkg{gbt7714} 宏包的 \file{-numeric} 样式后,
\cs{cite} 命令自动生成这种格式。

\begin{citeexample}
  第三系褶皱的控制\textsuperscript{[235]}
\end{citeexample}

同一处引用多篇文献时,将各篇文献的 key 一同写在 \cs{cite} 命令中,
如 |\cite{key1,key2,key3}|。
默认情况下,序号按照 \cs{cite} 命令中的顺序排列。

有时由于重复引用,序号不一定按照从小到大的顺序排列。
实际上,国标是允许这样引用的。
用户如果需要将引用的序号从小到大排序,
可以在调用宏包时或使用 \cs{gbtbstsetup} 命令设置 \opt{sort} 选项。

\begin{citeexample}
  同一处引用多篇文献\textsuperscript{[3,1,2]}
\end{citeexample}

如遇以上连续序号,\pkg{gbt7714} 会自动在起讫序号间并用短横线连接。
\pkg{natbib} 默认三个或以上的连续序号才会压缩为起讫序号,
而国标的示例中是两个以上连续序号就要压缩。
\pkg{gbt7714} 修改了这点以满足国标的要求。
用户也可以使用 \opt{mincompress} 选项设置这一缩数量。

\begin{citeexample}
  连续序号\textsuperscript{[1-2,4,6-9]}
\end{citeexample}

多次引用统一文献时,国标要求在序号的方括号“[ ]”外著录引文页码。

\begin{citeexample}
  引文页码\textsuperscript{[2]260}
\end{citeexample}

在有些情况下,序号也可作为行文语句的组成部分\cite[341]{chen2015}。
作者可以使用 \cs{parencite} 命令将序号与正文平排,
引文页码默认仍然以上标的形式著录在序号的方括号外。
如果全文统一采用这种格式,可以在调用宏包或使用 \cs{gbtbstsetup}
命令设置 \opt{cite-style=numbers} 选项。

\begin{citeexample}
  式(7)的具体推导过程见文献 [6]。……按文献[10]\textsuperscript{15}提供的参数设计出样机。
\end{citeexample}

\begin{quote}
  \begin{tabular}{l@{\quad$\Rightarrow$\quad}l}
    |\cite{key}| & \textsuperscript{[21]}\\
    |\cite{key1,key2}| & \textsuperscript{[21, 32]}\\
    |\cite[260]{key}| & \textsuperscript{[21]260}\\
    |\parencite{key}| & [21]\\
  \end{tabular}
\end{quote}


\subsection{著者-出版年制}

著者-出版年制是以著者姓名和出版年份索引的参考文献标引体系。
\pkg{natbib} 宏包提供了 \cs{citep} 和 \cs{citet} 两个命令
分别用于括号内和行文中的引用,
前者将著者和年份一起置于括号“( \ \  )”内,
后者将著者作为行文的一部分,年份置于括号内。
用户如果调用了 \pkg{gbt7714} 的 \file{-authoryear} 样式,
\cs{cite} 命令等同于 \cs{citet}。

\begin{citeexample}
  (张三,1984),Doe(2012)提出……
\end{citeexample}

倘若正文中提及著者姓名,其后的括号在分开的位置,
可以使用 \cs{citeauthor} 命令著录著者姓名,
再使用 \cs{citeyearpar} 命令著录括号内的出版年。

\begin{citeexample}
  张三提出……模型(1984)
\end{citeexample}

多次引用同一文献时,国标要求在括号外以角标的形式著录引文页码。
用户可以设置 \opt{locator-inside-brackets} 选项控制引文页码是否在括号外
(见第 \ref{sec:options} 节)。

\begin{citeexample}
  (张三,1984)\textsuperscript{260}
\end{citeexample}

\begin{quote}
  \begin{tabular}{l@{\quad$\Rightarrow$\quad}l}
    |\cite{key}| & 张三(1984)\\
    |\citet{key}| & 张三(1984)\\
    |\citep{key}| & (张三,1984)\\
    |\citep[260]{key}| & (张三,1984)\textsuperscript{260}\\
    |\citep[参见][]{key}| & (参见张三,1984)\\
    |\citep[参见][260]{key}| & (参见张三,1984)\textsuperscript{260}\\
    |\citeauthor{key}| & 张三\\
    |\citeyearpar[260]{key}| & (1984)\textsuperscript{260}\\
  \end{tabular}
\end{quote}


\section{设置选项}

\pkg{gbt7714} 宏包提供了一些选项供用户设置正文中引用标注的格式细节。
用户可在调用宏包时以可选参数的形式设置,
也可以在调用宏包后使用 \cs{gbtbstsetup} 命令进行设置。
下面两种设置方式是等效。

\begin{latexcode}
  \usepackage[sort]{gbt7714}
  \gbtbstsetup{sort}
\end{latexcode}

使用 \cs{ustcpackage} 调用宏包时,
未识别的选项将传给 \pkg{natbib} 宏包处理。
所以,用户可在传入 \pkg{natbib} 的控制选项。

本节的剩余部分列举了可用的选项。

% \begin{function}{\citestyle}
%   \begin{syntax}
%     |\citestyle|\Arg{citation style}
%   \end{syntax}
% 可选:\opt{super}, \opt{numbers}, \opt{author-year}。
% 使用 \cs{bibliography} 选择参考文献表的样式时会自动设置对应的引用样式。
% 顺序编码制的引用标注默认使用上标式(\opt{super}),
% 如“张三\textsuperscript{[2]}提出”。
% 如果要使用正文模式,如“文献 [3] 中说明”,
% 可以使用 \cs{citestyle} 命令切换为数字式(\opt{numbers})。
% \begin{latexcode}
%   \citestyle{numbers}
% \end{latexcode}
% 著者-出版年制通常不需要修改引用样式。
% \end{function}

% \DescribeOption{sort\&compress}
% 同一处引用多篇文献时,应当将各篇文献的 key 一同写在 \cs{cite} 命令中。
% 如遇连续编号,默认会自动转为起讫序号并用短横线连接
% (见\pkg{natbib} 的 \opt{compress} 选项)。
% 如果要对引用的编号进行自动排序,需要在调用 \pkg{gbt7714} 时加
% \opt{sort\&compress} 参数,
% 这些参数会传给 \pkg{natbib} 处理。
% \begin{latexcode}
%   \usepackage[sort&compress]{gbt7714}
% \end{latexcode}
% 注意国标中要求 2 个或以上的连续编号用连接号,不同于 \pkg{natbib} 默认的 3 个或以上。
% 宏包中已经作了修改。

\begin{optionlist}

  \boolitem[false]{sort}

  国标没有规定正文中引用标注的序号必须按顺序排列。
  实际上,国标 9.2.1.3 节的示例“[570,83]”没有按照顺序列。
  所以 \pkg{gbt7714} 默认情况下不对引用的序号进行排序。
  不过,用户可以设置 \opt{sort} 选项将引用的序号从小到大排序。

  \begin{citeexample}[title={\kvopt{sort}{false}}]
    [4,1,5,3]
  \end{citeexample}

  \begin{citeexample}[title={\kvopt{sort}{true}}]
    [1,3-5]
  \end{citeexample}


  \optitem[2]{mincompress}{2, 3}

  默认设置同国标的示例一样,两个或以上的连续序号时,起讫序号用短横线连接。
  该选项可以恢复为 \pkg{natbib} 的原始设置:
  三个或以上的连续序号才用短横线连接。

  \begin{citeexample}[title={\kvopt{mincompress}{2}}]
    [1-2,4-6]
  \end{citeexample}

  \begin{citeexample}[title={\kvopt{mincompress}{3}}]
    [1,2,4-6]
  \end{citeexample}



  \boolitem[false]{locator-inside-brackets}

  国标要求引文页码在括号外以角标的形式著录。
  选项控制引文页码是否在括号内著录。

  \begin{citeexample}[title={\kvopt{locator-inside-brackets}{false}}]
    |\citep[20]{key}| \qquad [20]\textsuperscript{260} \qquad (张三,1984)\textsuperscript{260}
  \end{citeexample}

  \begin{citeexample}[title={\kvopt{locator-inside-brackets}{true}}]
    |\citep[p. 20]{key}| \qquad [20, p. 260] \qquad (张三,1984,p. 260)
  \end{citeexample}


  \optitem[super]{cite-style}{super, numbers}

  采用顺序编码制时,默认引用标注使用上标式(\opt{super}),
  也允许序号作为行文语句的组成部分。
  如果全文统一采用后者,可以设置 \opt{cite-style=numbers} 选项。

  \begin{citeexample}[title={\kvopt{cite-style}{super}}]
    第三系褶皱的控制\textsuperscript{[2]260}
  \end{citeexample}

  \begin{citeexample}[title={\kvopt{cite-style}{numbers}}]
    第三系褶皱的控制 [2]\textsuperscript{260}
  \end{citeexample}

\end{optionlist}


% 使用时需要注意以下几点:
% \begin{itemize}
%   \item \file{.bib} 数据库应使用 UTF-8 编码。
%   \item 使用著者-出版年制参考文献表时,中文的文献\emph{必须}
%         在 key 域填写作者姓名的拼音,才能按照拼音排序,
%         详见第~\ref{sec:sort}~节。
% \end{itemize}



% \section{著录项目}

% 由于国标中规定的著录项目多于 \BibTeX{} 的标准域,
% 必须新增一些著录项目(带 * 号),
% 这些新增的类型在设计时参考了 BibLaTeX,如 date 和 urldate。
% 本宏包支持的全部域如下:
% \begin{description}
%   \item[author]       主要责任者
%   \item[title]        题名
%   \item[mark*]        文献类型标识
%   \item[medium*]      文献载体标识
%   \item[translator*]  译者
%   \item[editor]       编辑
%   \item[organization] 组织(用于会议)
%   \item[booktitle]    图书题名
%   \item[series]       系列
%   \item[journal]      期刊题名
%   \item[edition]      版本
%   \item[address]      出版地
%   \item[publisher]    出版者
%   \item[school]       学校(用于 \texttt{@phdthesis})
%   \item[institution]  机构(用于 \texttt{@techreport})
%   \item[year]         出版年
%   \item[volume]       卷
%   \item[number]       期(或者专利号)
%   \item[pages]        引文页码
%   \item[date*]        更新或修改日期
%   \item[urldate*]     引用日期
%   \item[url]          获取和访问路径
%   \item[doi]          数字对象唯一标识符
%   \item[langid*]      语言
%   \item[key]          拼音(用于排序)
% \end{description}
% 不支持的 \BibTeX{} 标准著录项目有 annote, chapter, crossref, month, type。

% 本宏包默认情况下可以自动识别文献语言,并自动处理文献类型和文献载体标识,
% 但是在少数情况下需要用户手动指定,如:
% \begin{latexcode}
%   @misc{citekey,
%     langid = {japanese},
%     mark   = {Z},
%     medium = {DK},
%     ...
%   }
% \end{latexcode}
% 可选的语言有 english, chinese, japanese, russian。


% \section{文献列表的排序}
% \label{sec:sort}

% 国标规定参考文献表采用著者-出版年制组织时,各篇文献首先按文种集中,
% 然后按著者字顺和出版年排列;
% 中文文献可以按著者汉语拼音字顺排列,也可以按著者的笔画笔顺排列。
% 然而由于 \BibTeX{} 功能的局限性,无法自动获取著者姓名的拼音或笔画笔顺,
% 所以\emph{必须}在 bib 数据库中的 key 域手动录入著者姓名的拼音用于排序,如:
% \begin{latexcode}
%   @book{capital,
%     author = {马克思 and 恩格斯},
%     key    = {ma3 ke4 si1 & en1 ge2 si1},
%     ...
%   }
% \end{latexcode}

% 对于著者-出版年的样式,如果中文文献较多时更推荐使用 \pkg{biblatex} 宏包,
% 其后端 \file{biber} 可以自动处理中文按照拼音排序,无须手动填写拼音。


% \section{自定义样式}

% \BibTeX{} 对自定义样式的支持比较有限,
% 所以用户只能通过修改 \file{bst} 文件来修改文献列表的格式。
% 本宏包提供了一些接口供用户更方便地修改。

% 在 \file{bst} 文件开始处的 |load.config| 函数中,
% 有一组配置参数用来控制样式,表~\ref{tab:config} 列出了每一项的默认值和功能。
% 若变量被设为 |#1| 则表示该项被启用,设为 |#0| 则不启用。
% 默认的值是严格遵循国标的配置。

% \begin{table}[htb]
% \centering\small
% \caption{参考文献表样式的配置参数}
% \label{tab:config}
% \begin{tabular}{lcl}
%   \toprule
%   参数值                         & 默认值 & 功能                           \\
%   \midrule
%   uppercase.name                 & |#1|   & 将著者姓名转为大写             \\
%   max.num.authors                & |#3|   & 输出著者的最多数量             \\
%   year.after.author              & |#0|   & 年份置于著者之后               \\
%   period.after.author            & |#0|   & 著者和年份之间使用句点连接     \\
%   control.emph.booktitle              & |#0|   & 西文书籍名使用斜体             \\
%   control.sentence.case.title            & |#1|   & 将西文的题名转为 sentence case \\
%   link.title                     & |#0|   & 在题名上添加 url 的超链接      \\
%   title.in.journal               & |#1|   & 期刊是否显示标题               \\
%   show.patent.country            & |#0|   & 专利题名是否含国别             \\
%   space.before.mark              & |#0|   & 文献类型标识前是否有空格       \\
%   show.mark                      & |#1|   & 显示文献类型标识               \\
%   show.medium.type               & |#1|   & 显示文献载体标识               \\
%   component.part.label           & |"slash"| & 表示析出文献的符号,可选:|"in"|, |"none"| \\
%   italic.journal                 & |#0|   & 西文期刊名使用斜体             \\
%   link.journal                   & |#0|   & 在期刊题名上添加 url 的超链接      \\
%   show.missing.address.publisher & |#0|   & 出版项缺失时显示“出版者不详”   \\
%   space.before.pages             & |#1|   & 页码与前面的冒号之间有空格     \\
%   only.start.page                & |#0|   & 只显示起始页码                 \\
%   page.range.delimiter           & |"-"|  & 起止页码中的连接号             \\
%   show.urldate                   & |#1|   & 显示引用日期 urldate           \\
%   show.url                       & |#1|   & 显示 url                       \\
%   show.doi                       & |#1|   & 显示 DOI                       \\
%   show.preprint                  & |#1|   & 显示预印本信息                 \\
%   show.note                      & |#0|   & 显示 note 域的信息             \\
%   end.with.period                & |#1|   & 结尾加句点                     \\
%   lowercase.word.after.colon     & |#1|   & 将冒号后的单词变成小写 \\
%   \bottomrule
% \end{tabular}
% \end{table}

% 若用户需要定制更多内容,可以学习 \file{bst} 文件的语法并修改
% \cite{btxhak,ttb,tlc2},或者联系作者。


% \section{相关工作}

% TeX 社区也有其他关于 GB/T 7714 系列参考文献标准的工作。
% 2005 年吴凯\cite{wk2006}发布了基于 GB/T 7714—2005 的 \BibTeX{} 样式,
% 支持顺序编码制和著者出版年制两种风格。
% 李志奇\cite{lqz2013}发布了严格遵循 GB/T 7714—2005 的 BibLaTeX 的样式。
% 胡海星\cite{hhx2013}提供了另一个 \BibTeX{} 实现,
% 还给每行 bst 代码写了 java 语言注释。
% 沈周\cite{sz2016}基于 \pkg{biblatex-caspervector}\cite{vector2012} 进行修改,
% 以符合国标的格式。
% 胡振震发布了符合 GB/T 7714—2015 标准的 BibLaTeX 参考文献样式\cite{hzz2016},
% 并进行了比较完善的持续维护。



\section{\file{.bib} 条目数据说明}
\label{sec:bibfile}

本节列举了 2025 国标规定的 14 类文献所对应的 \file{.bib} 条目示例,
其中需要特别注意的点会进行说明。
国标的全部示例文献所对应的 \file{.bib} 见
\href{https://github.com/zepinglee/gbt7714-bibtex-style}{GitHub 仓库}
的
\href{https://github.com/zepinglee/gbt7714-bibtex-style/blob/master/gbt7714-examples.bib}{gbt7714-examples.bib}
文件。

如果作者使用 Zotero 等文献管理软件导出 \file{.bib} 数据库,
建议选择导出为 \pkg{biblatex} 格式。
\pkg{biblatex} 在 \BibTeX{} 标准的基础上添加了一些文献类型和字段,
比 \BibTeX{} 标准更加丰富。

国标中规定了 14 种参考文献类型,
表~\ref{tab:entry-types} 列举了 \file{bib} 数据库中对应的文献类型。
其中部分类型超出了 \BibTeX{} 标准的范围,
所以使用了 \pkg{biblatex} 新增的条目类型。
% 这些尽可能兼容 \BibTeX{} 和 \pkg{biblatex} 的标准类型,但是新增了若干文献类型(带 * 号)。

\begin{table}[htb]
  \centering\small
  \begin{threeparttable}[b]
    \caption{各文献类型对应的 \BibTeX{} 条目类型}
    \label{tab:entry-types}
    \begin{tabular}{clcl}
      \toprule
      章节 & 文献类型         & 标识代码 & \BibTeX{} entry type                  \\
      \midrule
      8.2  & 图书         & M        & \entrytype{book}                        \\
      8.3  & 图书的析出文献   & M        & \entrytype{incollection} 或 \entrytype{inbook}            \\
      8.4  & 连续出版物           & J        & \entrytype{periodical}\tnote{1}                 \\
      8.5  & 连续出版物的析出文献           & J        & \entrytype{article}                 \\
      8.5  & 报纸中析出的文献          & N        & \entrytype{article}, \field{entrysubtype = \{newspaper\}}                \\
      8.6  & 会议录           & C        & \entrytype{proceedings}                 \\
      8.6  & 会议录的析出文献 & C        & \entrytype{inproceedings} 或 \entrytype{conference} \\
      8.7  & 学位论文         & D        & \entrytype{mastersthesis} 或 \entrytype{phdthesis}  \\
      8.8  & 报告             & R        & \entrytype{techreport}                  \\
      8.9  & 标准             & S        & \entrytype{standard}\tnote{1}                   \\
      8.10 & 专利             & P        & \entrytype{patent}\tnote{1}                     \\
      8.11 & 网站、网页         & EB       & \entrytype{online}\tnote{1}                     \\
      8.12 & 档案             & A        & \entrytype{archive}\tnote{2}                    \\
      8.13 & 地图             & CM       & \entrytype{map}\tnote{1}                        \\
      8.14 & 数据集           & DS       & \entrytype{dataset}\tnote{1}                    \\
      8.15 & 预印本           & PP       & \entrytype{preprint}\tnote{2}                   \\
      \bottomrule
    \end{tabular}
    \begin{tablenotes}
      \item[1] \pkg{biblatex} 的条目类型。
      \item[2] \pkg{gbt7714} 新增的条目类型。
    \end{tablenotes}
  \end{threeparttable}
\end{table}

\subsection{图书}

图书的卷次推荐使用阿拉伯数字的方式填写在 \field{volume} 字段中,
\pkg{gbt7714} 的 \BibTeX{} 样式会自动将其转换像“第 4 卷”、“v.1”这样的格式。
其他格式的卷次需要将完整内容填写在字段中
如“第 1 册”、“美国卷”、“第二卷”。

图书的版本填写在 \field{edition} 字段,
同样推荐使用阿拉伯数字的方式填写,
本宏包会自动将其转换为“3 版”、“5th ed.”这样的格式。
其他格式的版本需要将完整内容填写在字段中,
如“新 1 版”、“明刻本”、“Rev. ed.”。

\begin{bibtexfile}
@book{gbt7714.8.2.2:5,
  author        = {扬奎斯特 and 萨金特},
  title         = {递归宏观经济理论},
  translator    = {杨斌 and 王忠玉 and 陈彦斌 and 严高剑},
  series        = {经济科学译库},
  edition       = {2},
  address       = {北京},
  publisher     = {中国人民大学出版社},
  year          = {2010},
  pages         = {798},
  langid        = {chinese},
}
\end{bibtexfile}


\subsection{图书中的析出文献}

\begin{bibtexfile}
@book{gbt7714.8.2.2:1,
  author        = {张伯伟},
  title         = {全唐五代诗格汇考},
  address       = {南京},
  publisher     = {江苏古籍出版社},
  year          = {2002},
  pages         = {288},
  isbn          = {978-7-80643-578-6},
  langid        = {chinese}
}
\end{bibtexfile}


\subsection{连续出版物}

连续出版物的“年卷期或其他标识”著录在 \field{volume} 字段中。

\begin{bibtexfile}
@periodical{gbt7714.8.4.2:1,
  editor        = {中华医学会湖北分会},
  title         = {临床内科杂志},
  volume        = {1984, 1(1)-},
  address       = {武汉},
  publisher     = {中华医学会湖北分会},
  year          = {1984-},
  langid        = {chinese}
}
\end{bibtexfile}


\subsection{连续出版物中的析出文献}

\begin{bibtexfile}
@article{gbt7714.8.5.3:3,
  author        = {于潇 and 刘义 and 柴跃廷 and 孙宏波},
  title         = {互联网药品可信交易环境中主体资质审核备案模式},
  journal       = {清华大学学报(自然科学版)},
  year          = {2012},
  volume        = {52},
  number        = {11},
  pages         = {1518--1523},
  langid        = {chinese}
}
\end{bibtexfile}

报纸使用 \entrytype{article} 类型,
但需要著录 \field{entrysubtype = \{newspaper\}} 字段以区分报纸和期刊文章,
并且将出版日期和版次分别填写在 \field{date} 和 \field{pages} 字段中。

\begin{bibtexfile}
@article{gbt7714.8.5.3:1,
  entrysubtype  = {newspaper},
  author        = {丁文详},
  title         = {数字革命与竞争国际化},
  journal       = {中国青年报},
  date          = {2000-11-20},
  pages         = {15},
  langid        = {chinese}
}
\end{bibtexfile}

在线出版(或称为“网络首发”、“advance online publication”)的文章没有卷次和期号,
但需要著录出版日期(\field{date})。
\begin{bibtexfile}
@article{gbt7714.8.5.3:4,
  author        = {张群 and 程志宝 and 石志飞},
  title         = {惯性增强动力吸振器-浮置板轨道低频减振性能研究},
  journal       = {铁道学报},
  date          = {2024-05-09},
  urldate       = {2025-02-28},
  url           = {http://kns.cnki.net/kcms/detail/11.2104.u.20240507.1737.002.html},
  langid        = {chinese}
}
\end{bibtexfile}

文章编号著录在 \field{eid} 字段中。
\begin{bibtexfile}
@article{gbt7714.8.5.3:7,
  author        = {王利平 and 王福新 and 刘洪},
  title         = {过冷大水滴环境粒径分布模拟方法研究进展},
  journal       = {航空学报},
  year          = {2024},
  volume        = {45},
  number        = {增刊1},
  eid           = {730570},
  langid        = {chinese}
}
\end{bibtexfile}


\subsection{会议录}

以图书形式出版的会议录,使用 \entrytype{proceedings} 类型,其余与图书一致。
\begin{bibtexfile}
@proceedings{gbt7714.8.6.1:1,
  editor        = {牛志明 and Swingland, Ian R. and 雷光春},
  title         = {综合湿地管理:综合湿地管理国际研讨会论文集},
  address       = {北京},
  publisher     = {海洋出版社},
  year          = {2012},
  langid        = {chinese}
}
\end{bibtexfile}

以图书中的析出文献出现的会议论文,使用 \entrytype{inproceedings} 类型,
其余与 \entrytype{incollection} 一致。

\begin{bibtexfile}
@inproceedings{gbt7714.8.6.1:2,
  author        = {汪学军},
  title         = {中国农业转基因生物研发进展与安全管理},
  editor        = {国家环境保护总局生物安全管理办公室},
  booktitle     = {中国国家生物安全框架实施国际合作项目研讨会论文集},
  address       = {北京},
  publisher     = {中国环境科学出版社},
  year          = {2005},
  pages         = {22--25},
  langid        = {chinese}
}
\end{bibtexfile}

其余形式的会议论文将会议名称著录在 \field{eventtitle} 字段中。

\begin{bibtexfile}
@inproceedings{gbt7714.8.6.3:1,
  author        = {李妍 and 王莹},
  title         = {医疗机构保洁人员“一前五后”手卫生干预效果研究},
  eventtitle    = {中华预防医学会医院感染控制分会第31次全国医院感染学术年会},
  year          = {2022},
  pages         = {2},
  langid        = {chinese}
}
\end{bibtexfile}


\subsection{学位论文}

\begin{bibtexfile}
@phdthesis{gbt7714.8.7.2:1,
  author        = {王琦},
  title         = {融合星载GNSS-R和SAR数据的高时空分辨率土壤湿度反演方法研究},
  type          = {博士学位论文},
  address       = {武汉},
  school        = {武汉大学},
  year          = {2022},
  pages         = {87},
  langid        = {chinese}
}
\end{bibtexfile}


\subsection{报告}

报告需要将编号和日期分别填写在 \field{number} 和 \field{date} 字段中。

\begin{bibtexfile}
@techreport{gbt7714.8.8.3:1,
  author        = {汤万金 and 杨跃翔 and 刘文 and 郑建国 and 王赟松},
  title         = {人体安全重要技术标准研制最终报告},
  number        = {7178999X-2006BAK04A10/10.2013},
  date          = {2013-09-30},
  urldate       = {2014-06-24},
  url           = {http://www.nstrs.cn/xiangxiBG.aspx?id=41707},
  langid        = {chinese}
}
\end{bibtexfile}


\subsection{标准}

\begin{bibtexfile}
@standard{gbt7714.8.9.2:1,
  author        = {全国信息与文献标准化技术委员会},
  title         = {信息与文献\quad 资源描述},
  number        = {GB/T 3792—2021},
  year          = {2021},
  langid        = {chinese}
}
\end{bibtexfile}


\subsection{专利}

国标要求专利文献优先使用申请者/所有者(\field{holder}),而非发明者。
另外,应将专利申请号填写在 \field{number} 字段,而不是专利的公开号。
但是日期应使用公告/公开日期,不是申请日期。

2005 版国标要求著录专利国别,可以填写在 \field{address} 字段中。
最新版取消了该著录项目,这可能是因为专利申请号均带有国家代码前缀(如 “CN”),
没有必要再单独著录国别。

\begin{bibtexfile}
@patent{gbt7714.8.10.2:2,
  author        = {石顺祥 and 许海平 and 孙艳玲 and 陈利菊 and 李家立 and 刘继芳},
  holder        = {西安电子科技大学},
  title         = {光折变自适应光外差探测方法},
  address       = {中国},
  number        = {CN01128777.2},
  date          = {2002-03-06},
  urldate       = {2002-05-28},
  url           = {http://211.152.9.47/sipoasp/zljs/hyjs-yx-new.asp?recid=01128777.2&leixin=0},
  langid        = {chinese}
}
\end{bibtexfile}


\subsection{网站和网页}

\begin{bibtexfile}
@online{gbt7714.8.11.2.2:1,
  title         = {中国国家博物馆},
  urldate       = {2025-05-06},
  url           = {https://www.chnmuseum.cn/},
  langid        = {chinese}
}
\end{bibtexfile}


\subsection{档案}

档案文献使用 \entrytype{archive} 类型,
档号和收藏者分别填写在 \field{number} 和 \field{institution} 字段。
书信类文献也可以使用 \entrytype{letter} 类型。

\begin{bibtexfile}
@archive{gbt7714.8.12.3:2,
  author        = {湖北省建设厅},
  title         = {湖北省建设厅关于检发实业部农工矿业团体登记规则的布告、训令及湖北省政府的训令},
  number        = {LS031-001-0001-001},
  address       = {武汉},
  institution   = {湖北省档案馆},
  date          = {1931-11-07},
  urldate       = {2025-02-28},
  url           = {https://www.hbda.gov.cn/pdf/LS031-001-0001-001.PDF.PDF},
  langid        = {chinese}
}

@letter{gbt7714.8.12.3:4,
  author        = {Fitzwilliam, Hugh},
  recipient     = {{Bess of Hardwick}},
  title         = {[{Letter} to {Bess of Hardwick}]},
  date          = {1570-07-28},
  urldate       = {2025-02-28},
  url           = {https://www.bessofhardwick.org/letter.jsp?letter=25},
  langid        = {british}
}
\end{bibtexfile}


\subsection{地图}

地图的比例尺和尺寸分别填写在 \field{scale} 和 \field{dimensions} 字段。

\begin{bibtexfile}
@map{gbt7714.8.13.3:1,
  author        = {胡健民},
  title         = {东南极拉斯曼丘陵地区地质图},
  scale         = {1:25000},
  address       = {北京},
  publisher     = {科学出版社},
  year          = {2021},
  dimensions    = {128cm×84cm},
  langid        = {chinese}
}
\end{bibtexfile}


\subsection{数据集}

数据集的版本填写在 \field{version} 字段,不带前缀“V”。

\begin{bibtexfile}
@dataset{gbt7714.8.14.3:1,
  author        = {周壮 and 李盛阳 and 吴薇 and 郭威龙 and 李轩 and 夏桂松 and 赵子飞},
  title         = {天宫二号遥感图像自然景物分类数据集},
  version       = {1.0},
  publisher     = {国家基础学科公共科学数据中心},
  year          = {2023},
  date          = {2023-09-10},
  urldate       = {2025-07-15},
  url           = {https://www.nbsdc.cn/general/dataLinks/CSTR:16666.11.nbsdc.tfpbwtqf},
  cstr          = {16666.11.nbsdc.tfpbwtqf},
  langid        = {chinese}
}
\end{bibtexfile}


\subsection{预印本}

预印本推荐使用 \entrytype{preprint} 类型。
出版平台填写在 \field{archiveprefix} 字段中,

\begin{bibtexfile}
@preprint{gbt7714.8.15.2:1,
  author        = {肖玲 and 张雪 and 王永},
  title         = {数据要素的统计测算方法探究},
  archiveprefix = {PSSXiv},
  eprint        = {202408.01096},
  date          = {2024-07-02},
  urldate       = {2024-09-30},
  url           = {https://zsyyb.cn/abs/202408.01096},
  cstr          = {32012.36.PSSXiv.202408.01096},
  langid        = {chinese}
}
\end{bibtexfile}

\pkg{gbt7714} 同时兼容 arXiv 和 Google Scholar 导出的预印本格式,

\begin{bibtexfile}
@misc{jenkins2012controlledmanipulationlightcooperative,
      title={Controlled manipulation of light by cooperative response of atoms in an optical lattice},
      author={Stewart D. Jenkins and Janne Ruostekoski},
      year={2012},
      eprint={1112.6136},
      archivePrefix={arXiv},
      primaryClass={physics.optics},
      doi={https://doi.org/10.1103/PhysRevA.86.031602},
      url={https://arxiv.org/abs/1112.6136},
}
\end{bibtexfile}



\section{控制选项}
\label{sec:options}

本宏包提供了设置选项用于控制参考文献列表的格式细节。
这主要是为了满足不同期刊/学校在国标参考文献格式基础上的修改。
用户需要使用 \cs{gbtbstsetup} 命令来设置选项。
本节详细介绍各个选项的功能和用法。

\begin{latexcode}
\gbtbstsetup{
  punctstyle      = 2025,
  maxbibnames     = 3,
  uppercasefamily = false,
}
\end{latexcode}

目前 BibTeX 没有提供传入控制参数的接口,
无法将 LaTeX 中设置的格式选项直接传递给 BibTeX 样式(\file{.bst})并处理。
本宏包参考了 \pkg{IEEEtran} 和 \pkg{RevTeX} 等模板的做法,
修改后的 \cs{bibliography} 命令会自动生成一个带 \file{-BSTcontrol}
后缀的 \file{.bib} 文件,
并将格式控制选项以特殊类型“条目”(\entrytype{GBT7714BSTCTL})的形式写入该文件中。

\begin{bibtexfile}
@GBT7714BSTCTL{GBT7714:BSTcontrol,
  CTL_punct_style = {2025},
  CTL_max_bib_names = {3},
  CTL_uppercase_family = {false},
}
\end{bibtexfile}

宏包会在 \file{.aux} 调用 \file{-BSTcontrol.bib} 文件,
并“引用” \field{GBT7714:BSTcontrol},
这样 BibTeX 就能读取该条目的内容。
本宏包的 \file{.bst} 样式会根据 \field{GBT7714:BSTcontrol} 的字段信息设置内部参数。


\newcommand*{\highlighta}[1]{\textcolor{highlight1}{#1}}
\newcommand*{\highlightb}[1]{\textcolor{highlight2}{#1}}

\ExplSyntaxOn
\providecommand \gbtbstsetup [1] { \keys_set:nn { gbt7714 } { #1 } }
\ExplSyntaxOff


\begin{optionlist}

% \boolitem[true]{articlein}
% Whether or not to display \enquote{in:} before the journal information in
% \bibtype{article} entries.
% All other entry types are not affected by this option.
% If it is desired to remove the \enquote{in:} for more entry types or
% a more specific behaviour is required, then it is still going to be
% necessary to modify the bibmacro~\bibmacro{in:}.

% \begin{bibexample}[title={\kvopt{articlein}{true}}]
% 杨洪升.
% \newblock 四库馆私家抄校书考略\allowbreak[J].
% \newblock 文献,2013\allowbreak(1):56-75.
% \end{bibexample}

% \begin{bibexample}[title={\kvopt{articlein}{false}}]
% Bar
% \end{bibexample}

\subsection{全角/半角符号}
\label{sec:opt-punctstyle}

\optitem[2025]{punctstyle}{\opt{2025}, \opt{2015}, \opt{halfwidth}, \opt{bylanguage}}
选择半角/全角符号。
2025 版国标的示例是:“.”“[ ]”用半角符号,其他标识符号用全角符号。

实践中,参考文献的符号分为以下流派:
\setlength{\labelsep}{1em}
\begin{description}
  \item[\opt{2025}] “.”“[ ]”用半角符号,其他标识符号用全角符号。
  \item[\opt{2015}] 除“.”外,均使用中文符号,包括方括号。\cite{chen2024dui}
    % 包括方括号“[”(U+FF3B)、“]”(U+FF3D)\cite{chen2024dui}。
  \item[\opt{halfwidth}] 统一使用半角符号(例如清华大学学位论文)。
  \item[\opt{bylanguage}] 中文文献使用全角符号(包括全角句点 U+FF0E“.”)、西文文献使用半角符号。
    人文社科类期刊更多采用这种格式。
\end{description}

\begin{bibexample}[title={\kvopt{punctstyle}{2025}}]
  \gbtbstsetup{punctstyle = 2025}
  \bibitem[张群\ 等(2024)张群,程志宝和石志飞]{gbt7714.8.5.3:5}
  张群\highlightb{,}程志宝\highlightb{,}石志飞\highlighta{.}
  \newblock 惯性增强动力吸振器-浮置板轨道低频减振性能研究\allowbreak\highlighta{[}J\highlighta{]}\highlighta{.}
  \newblock 铁道学报\highlightb{,}2024\highlightb{,}46\allowbreak\highlightb{(}8\highlightb{)}\highlightb{:}102-111\highlighta{.}

  \bibitem[Frese et~al.(2013)Frese,Katus,and Meder]{gbt7714.b.4:16}
  Frese K~S\highlightb{,}Katus H~A\highlightb{,}Meder B\highlighta{.}
  \newblock Next-generation sequencing\highlightb{:}from understanding biology
    to personalized medicine\allowbreak\highlighta{[}J\highlighta{]}\highlighta{.}
  \newblock Biology\highlightb{,}2013\highlightb{,}2\allowbreak \highlightb{(}1\highlightb{)}\highlightb{:}378-398\highlighta{.}
\end{bibexample}

\begin{bibexample}[title={\kvopt{punctstyle}{2015}}]
  \gbtbstsetup{punctstyle = 2015}
  \bibitem[张群\ 等(2024)张群,程志宝和石志飞]{gbt7714.8.5.3:5}
  张群\highlightb{,}程志宝\highlightb{,}石志飞\highlighta{.}
  \newblock 惯性增强动力吸振器-浮置板轨道低频减振性能研究\allowbreak\highlightb{[}J\highlightb{]}\highlighta{.}
  \newblock 铁道学报\highlightb{,}2024\highlightb{,}46\allowbreak\highlightb{(}8\highlightb{)}\highlightb{:}102-111\highlighta{.}

  \bibitem[Frese et~al.(2013)Frese,Katus,and Meder]{gbt7714.b.4:16}
  Frese K~S\highlightb{,}Katus H~A\highlightb{,}Meder B\highlighta{.}
  \newblock Next-generation sequencing\highlightb{:}from understanding biology
    to personalized medicine\allowbreak\highlightb{[}J\highlightb{]}\highlighta{.}
  \newblock Biology\highlightb{,}2013\highlightb{,}2\allowbreak \highlightb{(}1\highlightb{)}\highlightb{:}378-398\highlighta{.}
\end{bibexample}

\begin{bibexample}[title={\kvopt{punctstyle}{halfwidth}}]
  \gbtbstsetup{punctstyle = halfwidth}
  \bibitem[张群\ 等(2024)张群,程志宝和石志飞]{gbt7714.8.5.3:5}
  张群\highlighta{,} 程志宝\highlighta{,} 石志飞\highlighta{.}
  \newblock 惯性增强动力吸振器-浮置板轨道低频减振性能研究\allowbreak\highlighta{[}J\highlighta{]}\highlighta{.}
  \newblock 铁道学报\highlighta{,} 2024\highlighta{,} 46\allowbreak\highlighta{(}8\highlighta{):} 102-111\highlighta{.}

  \bibitem[Frese et~al.(2013)Frese, Katus, and Meder]{gbt7714.b.4:16}
  Frese K~S\highlighta{,} Katus H~A\highlighta{,} Meder B\highlighta{.}
  \newblock Next-generation sequencing\highlighta{:} from understanding biology
    to personalized medicine\allowbreak\highlighta{[}J\highlighta{]}\highlighta{.}
  \newblock Biology\highlighta{,} 2013\highlighta{,} 2\allowbreak \highlighta{(}1\highlighta{)}\highlighta{:} 378-398\highlighta{.}
\end{bibexample}

\begin{bibexample}[title={\kvopt{punctstyle}{bylanguage}}]
  \bibitem[张群\ 等(2024)张群,程志宝和石志飞]{gbt7714.8.5.3:5}
  张群\highlightb{,}程志宝\highlightb{,}石志飞\highlightb{.}
  \newblock 惯性增强动力吸振器-浮置板轨道低频减振性能研究\allowbreak\highlightb{[}J\highlightb{]}\highlightb{.}
  \newblock 铁道学报\highlightb{,}2024\highlightb{,}46\allowbreak\highlightb{(}8\highlightb{)}\highlightb{:}102-111\highlightb{.}

  \bibitem[Frese et~al.(2013)Frese, Katus, and Meder]{gbt7714.b.4:16}
  Frese K~S\highlighta{,} Katus H~A\highlighta{,} Meder B\highlighta{.}
  \newblock Next-generation sequencing\highlighta{:} from understanding biology
    to personalized medicine\allowbreak\highlighta{[}J\highlighta{]}\highlighta{.}
  \newblock Biology\highlighta{,} 2013\highlighta{,} 2\allowbreak \highlighta{(}1\highlighta{)}\highlighta{:} 378-398\highlighta{.}
\end{bibexample}

\subsection{著录著者数量}

\intitem[3]{maxbibnames}
\intitem[3]{minbibnames}
参考文献表中如果责任者的姓名数量超过 \opt{maxbibnames},则会根据 \opt{minbibnames}
选项的设置自动截断列表,其后加“,等”或与之相应的词。
如果 \opt{maxbibnames} 为 0 则全部照录不省略。
默认为国标要求:超过 3 个时,著录前 3 个责任者。

\begin{bibexample}[title={\kvopt{maxbibnames}{3}, \kvopt{minbibnames}{3}}]
  \bibitem[牛永敢\ 等(2019)牛永敢,孔晓,王阳和斯楼斌]{gbt7714.8.2.2:8}
  牛永敢,孔晓,王阳,\highlighta{等}.
  \newblock 鼻整形应用解剖学\allowbreak[M].
  \newblock 北京:人民卫生出版社,2019:65-66.

  \bibitem[Sadock et~al.(2009)Sadock,Sadock,Ruiz,and Kaplan]{gbt7714.8.2.2:14}
  Sadock B~J,Sadock V~A,Ruiz P,\highlighta{et~al.}
  \newblock Kaplan \& {Sadock's} comprehensive textbook of psychiatry:v.~1\allowbreak[M].
  \newblock 9th ed.
  \newblock Philadelphia:Wolters Kluwer Health/Lippincott Williams \& Wilkins,2009.
\end{bibexample}

\begin{bibexample}[title={\kvopt{maxbibnames}{0}, \kvopt{minbibnames}{0}}]
  \bibitem[牛永敢\ 等(2019)牛永敢,孔晓,王阳和斯楼斌]{gbt7714.8.2.2:8}
  牛永敢,孔晓,王阳,\highlightb{斯楼斌}.
  \newblock 鼻整形应用解剖学\allowbreak[M].
  \newblock 北京:人民卫生出版社,2019:65-66.

  \bibitem[Sadock et~al.(2009)Sadock,Sadock,Ruiz,and Kaplan]{gbt7714.8.2.2:14}
  Sadock B~J,Sadock V~A,Ruiz P,\highlightb{Kaplan H~I}.
  \newblock Kaplan \& {Sadock's} comprehensive textbook of psychiatry:v.~1\allowbreak[M].
  \newblock 9th ed.
  \newblock Philadelphia:Wolters Kluwer Health/Lippincott Williams \& Wilkins,2009.
\end{bibexample}

\intitem[1]{maxcitenames}
\intitem[1]{mincitenames}
正文引用的文献采用著者-出版年制时,如果责任者的姓名数量超过 \opt{maxcitenames},则会根据 \opt{mincitenames}
选项的设置自动截断列表,其后加“等”或与之相应的词。
如果 \opt{maxcitenames} 为 0 则全部照录不省略。
默认为国标要求:正文中引用多著者文献时,只需标注第一个著者。

\begin{citeexample}[title={\kvopt{maxcitenames}{1}, \kvopt{mincitenames}{1}}]
  (战德臣~\highlighta{等},2019)(Kennedy \highlighta{et al.},1975)
\end{citeexample}

\begin{citeexample}[title={\kvopt{maxbibnames}{2}, \kvopt{minbibnames}{1}}]
  (战德臣\highlightb{和张丽杰},2019)(Kennedy \highlightb{and Garrison},1975)
\end{citeexample}


\subsection{姓全大写}

\boolitem[false (2025), true (2015)]{uppercasefamily}

控制姓氏是否全大写。

2005 和 2015 版国标要求:用西文或汉语拼音字母著录个人作者,姓应全部著录,字母全大写。
2025 版取消全大写的要求。

\begin{bibexample}[title={\kvopt{uppercasefamily}{false}}]
  \bibitem[Peebles(2001)]{gbt7714.8.2.2:13}
  \highlighta{Peebles} P~Z,Jr.
  \newblock Probability,random variables,and random signal principles\allowbreak[M].
  \newblock 4th ed.
  \newblock New York:McGraw-Hill,2001.
\end{bibexample}

\begin{bibexample}[title={\kvopt{uppercasefamily}{true}}]
  \bibitem[Peebles(2001)]{gbt7714.8.2.2:13}
  \highlightb{PEEBLES} P~Z,Jr.
  \newblock Probability,random variables,and random signal principles\allowbreak[M].
  \newblock 4th ed.
  \newblock New York:McGraw-Hill,2001.
\end{bibexample}


\subsection{姓名使用“和”或“and”}

\boolitem[false]{bibfinaland}
控制参考文献表中多个责任者姓名的最后两个姓名之间使用“和”或“and”连接。

\begin{bibexample}[title={\kvopt{bibfinaland}{false}}]
  \bibitem[张群\ 等(2024)张群,程志宝,石志飞]{gbt7714.8.5.3:5}
  张群,程志宝\highlighta{,}石志飞.
  \newblock 惯性增强动力吸振器-浮置板轨道低频减振性能研究\allowbreak[J].
  \newblock 铁道学报,2024,46\allowbreak (8):102-111.

  \bibitem[Abadía et~al.(2024)Abadía,Conkey,and McDonald]{gbt7714.8.2.2:8}
  Abadía O~M,Conkey M~W,McDonald J.
  \newblock Deep-time images in the age of globalization:rock art in the 21st century\allowbreak[M/OL].
  \newblock Springer Cham,2024.
  \newblock \url{https://doi.org/10.1007/978-3-031-54638-9}.
\end{bibexample}

\begin{bibexample}[title={\kvopt{bibfinaland}{true}}]
  \bibitem[张群\ 等(2024)张群,程志宝和石志飞]{gbt7714.8.5.3:5}
  张群,程志宝\highlightb{和}石志飞.
  \newblock 惯性增强动力吸振器-浮置板轨道低频减振性能研究\allowbreak[J].
  \newblock 铁道学报,2024,46\allowbreak (8):102-111.

  \bibitem[Abadía et~al.(2024)Abadía,Conkey,and McDonald]{gbt7714.8.2.2:8}
  Abadía O~M,Conkey M~W,\highlightb{and} McDonald J.
  \newblock Deep-time images in the age of globalization:rock art in the 21st century\allowbreak[M/OL].
  \newblock Springer Cham,2024.
  \newblock \url{https://doi.org/10.1007/978-3-031-54638-9}.
\end{bibexample}

\boolitem[true]{citefinaland}
控正文引用中多个责任者姓名的最后两个姓名之间使用“和”或“and”连接。
注意需要将 \opt{maxcitenames} 设置为大于 1 的值才能看到效果。
该选项仅对著者-出版年制有效。

\begin{citeexample}[title={\kvopt{citefinaland}{true}, \kvopt{maxcitenames}{2}}]
  (扬奎斯特\highlighta{和}萨金特,2010)(Weinstein \highlighta{and} Swartz,1974)
\end{citeexample}

\begin{citeexample}[title={\kvopt{citefinaland}{false}, \kvopt{maxcitenames}{2}}]
  (扬奎斯特\highlightb{,}萨金特,2010)(Weinstein\highlightb{,}Swartz,1974)
\end{citeexample}


\subsection{引注的本地化字符串}

\optitem[auto]{citelang}{auto, macro, chinese, english}
控制引注中“and”、“et al.”的语言。
\begin{description}
  \item[\opt{auto}] 根据文献的语言自动选择。
  \item[\opt{macro}] 使用宏 \cs{biband}、\cs{bibetal},供用户自定义。
  \item[\opt{chinese}] 总是使用中文。
  \item[\opt{english}] 总是使用英文。
\end{description}

\begin{citeexample}[title={\kvopt{citelang}{auto}, \kvopt{maxcitenames}{2}}]
  (战德臣\highlighta{和}张丽杰,2019)
  (牛永敢~\highlighta{等},2019) \\
  (Kennedy \highlighta{and} Garrison,1975)
  (Sadock \highlighta{et~al.}, 2009)
\end{citeexample}

\begin{citeexample}[title={\kvopt{citelang}{chinese}, \kvopt{maxcitenames}{2}}]
  (战德臣\highlighta{和}张丽杰,2019)
  (牛永敢~\highlighta{等},2019) \\
  (Kennedy \highlightb{和} Garrison,1975)
  (Sadock \highlightb{等}, 2009)
\end{citeexample}

\begin{citeexample}[title={\kvopt{citelang}{english}, \kvopt{maxcitenames}{2}}]
  (战德臣 \highlightb{and} 张丽杰,2019)
  (牛永敢 \highlightb{et al.},2019) \\
  (Kennedy \highlighta{and} Garrison,1975)
  (Sadock \highlighta{et~al.}, 2009)
\end{citeexample}


\subsection{姓名与“等”的空隙}

\boolitem[true]{spacebeforeetal}
国标要求著者-出版年制正文的引注中,姓氏与“et al.”“等”之间留适当空隙。
该选项可以取消中文文献的“等”字前空隙,不影响英文的“et al.”。

\begin{citeexample}[title={\kvopt{spacebeforeetal}{true}}]
  (于潇~\highlighta{等},2012)(Abadía~\highlighta{et al.},2024)
\end{citeexample}

\begin{citeexample}[title={\kvopt{spacebeforeetal}{false}}]
  (于潇\highlightb{等},2012)(Abadía~\highlighta{et al.},2025)
\end{citeexample}


\subsection{出版年置于题名前}

\boolitem[true]{yearbeforetitle}
采用著者-出版年制时,参考文献表中出版年应置于题名前,
区别于顺序编码制。
然而实践中仍有不少期刊/学校要求将出版年置于题名之后。

该选项不影响顺序编码制。

\begin{authoryearbibexample}[title={\kvopt{yearbeforetitle}{true}}]
  \bibitem[杨宗英(1996)]{gbt7714.9.2.2:5}
  杨宗英,\highlighta{1996}.
  \newblock 电子图书馆的现实模型\allowbreak[J].
  \newblock 中国图书馆学报\allowbreak(2):24-29.

  \bibitem[Baker et~al.(1995)Baker and Jackson]{gbt7714.9.2.2:1}
  Baker S~K,Jackson M~E,\highlighta{1995}.
  \newblock The future of resource sharing\allowbreak[M].
  \newblock New York:The Haworth Press.
\end{authoryearbibexample}

\begin{authoryearbibexample}[title={\kvopt{yearbeforetitle}{false}}]
  \bibitem[杨宗英(1996)]{gbt7714.9.2.2:5}
  杨宗英.
  \newblock 电子图书馆的现实模型\allowbreak[J].
  \newblock 中国图书馆学报,\highlightb{1996}\allowbreak(2):24-29.

  \bibitem[Baker et~al.(1995)Baker and Jackson]{gbt7714.9.2.2:1}
  Baker S~K,Jackson M~E.
  \newblock The future of resource sharing\allowbreak[M].
  \newblock New York:The Haworth Press,\highlightb{1995}.
\end{authoryearbibexample}


\subsection{著者和出版年的分隔符}

\optitem[comma]{nameyeardelim}{comma, period}
采用著者-出版年制时,参考文献表中出版年前的符号自 2015 版国标改为逗号。
在著录实践中,仍存在不少使用句点的。
该选项不影响顺序编码制。

\begin{authoryearbibexample}[title={\kvopt{nameyeardelim}{comma}}]
  \bibitem[杨宗英(1996)]{gbt7714.9.2.2:5}
  杨宗英\highlighta{,}1996.
  \newblock 电子图书馆的现实模型\allowbreak[J].
  \newblock 中国图书馆学报\allowbreak(2):24-29.

  \bibitem[Baker et~al.(1995)Baker and Jackson]{gbt7714.9.2.2:1}
  Baker S~K,Jackson M~E\highlighta{,}1995.
  \newblock The future of resource sharing\allowbreak[M].
  \newblock New York:The Haworth Press.
\end{authoryearbibexample}

\begin{authoryearbibexample}[title={\kvopt{nameyeardelim}{period}}]
  \bibitem[杨宗英(1996)]{gbt7714.9.2.2:5}
  杨宗英\highlightb{.} 1996.
  \newblock 电子图书馆的现实模型\allowbreak[J].
  \newblock 中国图书馆学报\allowbreak(2):24-29.

  \bibitem[Baker et~al.(1995)Baker and Jackson]{gbt7714.9.2.2:1}
  Baker S~K,Jackson M~E\highlightb{.} 1995.
  \newblock The future of resource sharing\allowbreak[M].
  \newblock New York:The Haworth Press.
\end{authoryearbibexample}


\subsection{标题大小写}
\label{sec:opt-sentence-case}

\boolitem[true]{sentencecase}
国标的示例中均采用 sentence case 著录英文题名,即句首和专有名词首字母大写,其余小写。
这区别于另一种大小写风格 title case: 句首和实词首字母均大写。

在使用 BibTeX 时,正确的使用方法是在 \file{.bib} 文件中使用 title case,
并专有名词用大括号保护起来。
这样可以使 BibTeX 根据引注样式控制转换为 sentence case,
并且避免错误地将专有名词转换为小写。

\begin{bibtexfile}
@book{intro-latex,
  title = {An Introduction to {LaTeX}},
}
\end{bibtexfile}

选项 \opt{sentencecase} 可用统一控制题名、析出文献的图书题名是否转为 sentence case,
即同时设置 \opt{sentencecasetitle}、\opt{sentencecasebooktitle} 这两个选项。
使用 2005 或 2015 版国标时,该选项还会额外设置 \opt{sentencecasejournal} 值。

\boolitem[true]{sentencecasetitle}
控制题名(\field{title})是否转为 sentence case。

\begin{bibexample}[title={\kvopt{sentencecasetitle}{true}}]
  \bibitem[Peebles(2001)]{gbt7714.8.2.2:6}
  Peebles P~Z,Jr.
  \newblock \highlighta{Probability,random variable,and random signal principles}\allowbreak[M].
  \newblock 4th ed.
  \newblock New York:McGraw-Hill,2001.
\end{bibexample}

\begin{bibexample}[title={\kvopt{sentencecasetitle}{false}}]
  \bibitem[Peebles(2001)]{gbt7714.8.2.2:6}
  Peebles P~Z,Jr.
  \newblock \highlightb{Probability,Random Variable,and Random Signal Principles}\allowbreak[M].
  \newblock 4th ed.
  \newblock New York:McGraw-Hill,2001.
\end{bibexample}


\boolitem[true]{sentencecasebooktitle}
控制析出文献的图书题名(\field{booktitle})是否转为 sentence case。
图书题名包括会议论文集的题名,但不包括会议名称(\field{eventtitle})。

\begin{bibexample}[title={\kvopt{sentencecasebooktitle}{true}}]
  \bibitem[Weinstein et~al.(1974)Weinstein and Swartz]{gbt7714.8.3.2:4}
  Weinstein L,Swartz M~N.
  \newblock Pathogenic properties of invading microorganisms\allowbreak[M]//\allowbreak
  Sodeman W~A,Jr,Sodeman W~A.
  \newblock \highlighta{Pathologic physiology:mechanisms of disease}.
  \newblock 5th ed.
  \newblock Philadelphia:Saunders,1974:457-472.
\end{bibexample}

\begin{bibexample}[title={\kvopt{sentencecasebooktitle}{false}}]
  \bibitem[Weinstein et~al.(1974)Weinstein and Swartz]{gbt7714.8.3.2:4}
  Weinstein L,Swartz M~N.
  \newblock Pathogenic properties of invading microorganisms\allowbreak[M]//\allowbreak
  Sodeman W~A,Jr,Sodeman W~A.
  \newblock \highlightb{Pathologic Physiology:Mechanisms of Disease}.
  \newblock 5th ed.
  \newblock Philadelphia:Saunders,1974:457-472.
\end{bibexample}


\boolitem[false (2025), true (2015)]{sentencecasejournal}
控制刊名是否转为 sentence case。
使用 2025 版国标的样式时,该选项默认为 false。
如果使用 2005 或 2015 版国标样式,该选项默认为 true。

\begin{bibexample}[title={\kvopt{sentencecasejournal}{false}}]
  \bibitem[Caplan(1993)]{gbt7714.8.1:4}
  Caplan P.
  \newblock Cataloging internet resources\allowbreak[J].
  \newblock \highlighta{The Public-Access Computer Systems Review},1993,4\allowbreak (2):61-66.
\end{bibexample}

\begin{bibexample}[title={\kvopt{sentencecasejournal}{true}}]
  \bibitem[Caplan(1993)]{gbt7714.8.1:4}
  Caplan P.
  \newblock Cataloging internet resources\allowbreak[J].
  \newblock \highlightb{The public-access computer systems review},1993,4\allowbreak (2):61-66.
\end{bibexample}


\subsection{冒号后大小写}

\boolitem[false]{capitalizesubtitle}
国标示例中,英文文献题名的冒号后首字母小写。
部分国外体例(如 APA、IEEE)要求冒号后首字母大写。
该选项仅影响转换为 sentence case 的题名(参考第~\ref{sec:opt-sentence-case} 节)。

\begin{bibexample}[title={\kvopt{capitalizesubtitle}{false}}]
  \bibitem[Abadía et~al.(2024)Abadía,Conkey,and McDonald]{gbt7714.8.2.2:8}
  Abadía O~M,Conkey M~W,McDonald J.
  \newblock Deep-time images in the age of globalization:\highlighta{rock} art in the 21st century\allowbreak[M/OL].
  \newblock Springer Cham,2024.
  \newblock \url{https://doi.org/10.1007/978-3-031-54638-9}.
\end{bibexample}

\begin{bibexample}[title={\kvopt{capitalizesubtitle}{true}}]
  \bibitem[Abadía et~al.(2024)Abadía,Conkey,and McDonald]{gbt7714.8.2.2:8}
  Abadía O~M,Conkey M~W,McDonald J.
  \newblock Deep-time images in the age of globalization:\highlightb{Rock} art in the 21st century\allowbreak[M/OL].
  \newblock Springer Cham,2024.
  \newblock \url{https://doi.org/10.1007/978-3-031-54638-9}.
\end{bibexample}


\subsection{期刊文章著录题名}

\boolitem[true]{articletitle}
控制是否著录期刊文章的题名(例如 APS 的样式)。
注意:该选项也会影响图书的析出文献(\entrytype{incollection})、
会议论文(\entrytype{inproceedings})的题名。

\begin{bibexample}[title={\kvopt{articletitle}{true}}]
  \bibitem[于潇\ 等(2012)于潇,刘义,柴跃廷和孙宏波]{gbt7714.8.5.3:3}
  于潇,刘义,柴跃廷,等.
  \newblock \highlighta{互联网药品可信交易环境中主体资质审核备案模式\allowbreak[J].}
  \newblock 清华大学学报(自然科学版),2012,52\allowbreak(11):1518-1523.

  \bibitem[Caplan(1993)]{gbt7714.8.5.3:10}
  Caplan P.
  \newblock \highlighta{Cataloging internet resources\allowbreak[J].}
  \newblock The Public-Access Computer Systems Review,1993,4\allowbreak(2):61-66.
\end{bibexample}

\begin{bibexample}[title={\kvopt{articletitle}{false}}]
  \bibitem[于潇\ 等(2012)于潇,刘义,柴跃廷和孙宏波]{gbt7714.8.5.3:3}
  于潇,刘义,柴跃廷,等.
  \newblock 清华大学学报(自然科学版),2012,52\allowbreak(11):1518-1523.

  \bibitem[Caplan(1993)]{gbt7714.8.5.3:10}
  Caplan P.
  \newblock The Public-Access Computer Systems Review,1993,4\allowbreak(2):61-66.
\end{bibexample}


\subsection{专利国别}

\boolitem[false]{patentcountry}
控制是否著录专利国别。

2025 版国标的专利示例中,专利号中均有国家代码前缀,似乎没有必要再著录专利国别。

% \begin{bibexample}[title={\kvopt{patentcountry}{false}}]
%   \bibitem[邓一刚(2008)]{gbt7714.8.10.2:1}
%   邓一刚.
%   \newblock 全智能节电器:CN200610171314.3\allowbreak[P].
%   \newblock 2008-01-16:8-9.
% \end{bibexample}

% \begin{bibexample}[title={\kvopt{patentcountry}{true}}]
%   \bibitem[邓一刚(2008)]{gbt7714.8.10.2:1}
%   邓一刚.
%   \newblock 全智能节电器:中国,CN200610171314.3\allowbreak[P].
%   \newblock 2008-01-16:8-9.
% \end{bibexample}


\subsection{文献类型标识}

\boolitem[true]{entrytypeid}
控制是否著录文献类型标识(含文献载体标识)。

\begin{bibexample}[title={\kvopt{entrytypeid}{true}}]
  \bibitem[张伯伟(2002)]{gbt7714.8.2.2:1}
  张伯伟.
  \newblock 全唐五代诗格汇考\allowbreak\highlighta{[M]}.
  \newblock 南京:江苏古籍出版社,2002:288.

  \bibitem[Myburg et~al.(2014)Myburg,Grattapaglia,Tuskan,Hellsten,Hayes,Grimwood,Jenkins,Lindquist,Tice,Bauer,Goodstein,Dubchak,Poliakov,Mizrachi,Kullan,Hussey,Pinard,van~der Merwe,Singh,van Jaarsveld,Silva-Junior,Togawa,Pappas,Faria,Sansaloni,Petroli,Yang,Ranjan,Tschaplinski,Ye,Li,Sterck,Vanneste,Murat,Soler,Clemente,Saidi,Cassan-Wang,Dunand,Hefer,Bornberg-Bauer,Kersting,Vining,Amarasinghe,Ranik,Naithani,Elser,Boyd,Liston,Spatafora,Dharmwardhana,Raja,Sullivan,Romanel,Alves-Ferreira,Külheim,Foley,Carocha,Paiva,Kudrna,Brommonschenkel,Pasquali,Byrne,Rigault,Tibbits,Spokevicius,Jones,Steane,Vaillancourt,Potts,Joubert,Barry,Pappas,Strauss,Jaiswal,Grima-Pettenati,Salse,Van~de Peer,Rokhsar,and Schmutz]{gbt7714.8.5.3:9}
  Myburg A~A,Grattapaglia D,Tuskan G~A,et~al.
  \newblock The genome of {{\em Eucalyptus grandis}}\allowbreak\highlighta{[J/OL]}.
  \newblock Nature,2014,510:356-362.
  \newblock \url{https://www.nature.com/articles/nature13308.pdf}.
  \newblock DOI:\doi{10.1038/nature13308}.
\end{bibexample}

\begin{bibexample}[title={\kvopt{entrytypeid}{false}}]
  \bibitem[张伯伟(2002)]{gbt7714.8.2.2:1}
  张伯伟.
  \newblock 全唐五代诗格汇考.
  \newblock 南京:江苏古籍出版社,2002:288.

  \bibitem[Myburg et~al.(2014)Myburg,Grattapaglia,Tuskan,Hellsten,Hayes,Grimwood,Jenkins,Lindquist,Tice,Bauer,Goodstein,Dubchak,Poliakov,Mizrachi,Kullan,Hussey,Pinard,van~der Merwe,Singh,van Jaarsveld,Silva-Junior,Togawa,Pappas,Faria,Sansaloni,Petroli,Yang,Ranjan,Tschaplinski,Ye,Li,Sterck,Vanneste,Murat,Soler,Clemente,Saidi,Cassan-Wang,Dunand,Hefer,Bornberg-Bauer,Kersting,Vining,Amarasinghe,Ranik,Naithani,Elser,Boyd,Liston,Spatafora,Dharmwardhana,Raja,Sullivan,Romanel,Alves-Ferreira,Külheim,Foley,Carocha,Paiva,Kudrna,Brommonschenkel,Pasquali,Byrne,Rigault,Tibbits,Spokevicius,Jones,Steane,Vaillancourt,Potts,Joubert,Barry,Pappas,Strauss,Jaiswal,Grima-Pettenati,Salse,Van~de Peer,Rokhsar,and Schmutz]{gbt7714.8.5.3:9}
  Myburg A~A,Grattapaglia D,Tuskan G~A,et~al.
  \newblock The genome of {{\em Eucalyptus grandis}}.
  \newblock Nature,2014,510:356-362.
  \newblock \url{https://www.nature.com/articles/nature13308.pdf}.
  \newblock DOI:\doi{10.1038/nature13308}.
\end{bibexample}


\subsection{文献类型标识前的空格}

\boolitem[false]{spacebeforetypeid}
控制题名与文献类型标识之间是否有空格。
国标的示例中均没有这一空格,但是按照西文的习惯,文献类型标识作为单独的词应该与题名分开。
该选项不影响中文文献。

\begin{bibexample}[title={\kvopt{spacebeforetypeid}{false}}]
  \bibitem[张伯伟(2002)]{gbt7714.8.2.2:1}
  张伯伟.
  \newblock 全唐五代诗格汇考\allowbreak\highlighta{[M]}.
  \newblock 南京:江苏古籍出版社,2002:288.

  \bibitem[Caplan(1993)]{gbt7714.8.1:4}
  Caplan P.
  \newblock Cataloging internet resources\allowbreak\highlighta{[J]}.
  \newblock The Public-Access Computer Systems Review,1993,4\allowbreak (2):61-66.
\end{bibexample}

\begin{bibexample}[title={\kvopt{spacebeforetypeid}{true}}]
  \bibitem[张伯伟(2002)]{gbt7714.8.2.2:1}
  张伯伟.
  \newblock 全唐五代诗格汇考\allowbreak\highlighta{[M]}.
  \newblock 南京:江苏古籍出版社,2002:288.

  \bibitem[Caplan(1993)]{gbt7714.8.1:4}
  Caplan P.
  \newblock Cataloging internet resources \highlightb{[J]}.
  \newblock The Public-Access Computer Systems Review,1993,4\allowbreak (2):61-66.
\end{bibexample}


\subsection{文献载体标识}

\boolitem[true]{entrymediumid}
控制是否著录文献载体标识。

国标的示例中,文献载体标识“/OL”都是与 URL 或永久标识符同时出现的。
当设置 \opt{url=false} 和 \opt{doi=false} 选项时,
除电子资源(如网页、数据集等)外自动不著录文献载体标识。

\begin{bibexample}[title={\kvopt{entrymediumid}{true}}]
  \bibitem[赵学功(2001)]{gbt7714.b.1:14}
  赵学功.
  \newblock 当代美国外交\allowbreak[M/OL].
  \newblock 北京:社会科学文献出版社,2001.
  \newblock \url{http://www.cadal.zju.edu.cn/book/trySinglePage/33023884/1}.

  \bibitem[Anon({[2020]})]{gbt7714.8.11.2.2:2}
  {Library of Congress}\allowbreak[EB/OL].
  \newblock \allowbreak[2020-06-12].
  \newblock \url{https://www.loc.gov/}.
\end{bibexample}

\begin{bibexample}[title={\kvopt{entrymediumid}{false}}]
  \bibitem[赵学功(2001)]{gbt7714.b.1:14}
  赵学功.
  \newblock 当代美国外交\allowbreak[M].
  \newblock 北京:社会科学文献出版社,2001.
  \newblock \url{http://www.cadal.zju.edu.cn/book/trySinglePage/33023884/1}.

  \bibitem[Anon({[2020]})]{gbt7714.8.11.2.2:2}
  {Library of Congress}\allowbreak[EB].
  \newblock \allowbreak[2020-06-12].
  \newblock \url{https://www.loc.gov/}.
\end{bibexample}


\subsection{析出文献分隔符}

\optitem[slash]{componentpartdelim}{slash, period}
控制论文集、会议录的析出文献出处(包括会议名称前)的分隔符。默认为 \opt{slash} “//”。

\begin{bibexample}[title={\kvopt{componentpartdelim}{slash}}]
  \bibitem[程根伟(1999)]{gbt7714.8.1:2}
  程根伟.
  \newblock 1998年长江洪水的成因与减灾对策\allowbreak[M]\highlighta{//}\allowbreak
  许厚泽,赵其国.
  \newblock 长江流域洪涝灾害与科技对策.
  \newblock 北京:科学出版社,1999:32-36.

  \bibitem[Wang(2022)]{gbt7714.8.6.3:2}
  Wang Shanshan.
  \newblock Application of improved {SOM} neural network in intelligent auditing of hospital financial vouchers\allowbreak[C/OL]\highlighta{//}\allowbreak
  2022 6th {Asian Conference} on {Artificial Intelligence Technology}({ACAIT}),2022:2.
  \newblock \url{https://ieeexplore.ieee.org/document/10137867}.
  \newblock DOI:\doi{10.1109/ACAIT56212.2022.10137867}.
\end{bibexample}

\begin{bibexample}[title={\kvopt{componentpartdelim}{period}}]
  \bibitem[程根伟(1999)]{gbt7714.8.1:2}
  程根伟.
  \newblock 1998年长江洪水的成因与减灾对策\allowbreak[M]\highlightb{.}
  \newblock 许厚泽,赵其国.
  \newblock 长江流域洪涝灾害与科技对策.
  \newblock 北京:科学出版社,1999:32-36.

  \bibitem[Wang(2022)]{gbt7714.8.6.3:2}
  Wang Shanshan.
  \newblock Application of improved {SOM} neural network in intelligent auditing of hospital financial vouchers\allowbreak[C/OL]\highlightb{.}
  \newblock 2022 6th {Asian Conference} on {Artificial Intelligence Technology}({ACAIT}),2022:2.
  \newblock \url{https://ieeexplore.ieee.org/document/10137867}.
  \newblock DOI:\doi{10.1109/ACAIT56212.2022.10137867}.
\end{bibexample}


\subsection{析出文献著录“见”或“In”}

\boolitem[false]{in}
控制论文集、会议录的析出文献出处(包括会议名称前)是否著录“见:”或“In:”。

\begin{bibexample}[title={\kvopt{in}{false}}]
  \bibitem[程根伟(1999)]{gbt7714.8.1:2}
  程根伟.
  \newblock 1998年长江洪水的成因与减灾对策\allowbreak[M]//\allowbreak
  许厚泽,赵其国.
  \newblock 长江流域洪涝灾害与科技对策.
  \newblock 北京:科学出版社,1999:32-36.

  \bibitem[Wang(2022)]{gbt7714.8.6.3:2}
  Wang Shanshan.
  \newblock Application of improved {SOM} neural network in intelligent auditing of hospital financial vouchers\allowbreak[C/OL]//\allowbreak
  2022 6th {Asian Conference} on {Artificial Intelligence Technology}({ACAIT}),2022:2.
  \newblock \url{https://ieeexplore.ieee.org/document/10137867}.
  \newblock DOI:\doi{10.1109/ACAIT56212.2022.10137867}.
\end{bibexample}

\begin{bibexample}[title={\kvopt{in}{true}}]
  \bibitem[程根伟(1999)]{gbt7714.8.1:2}
  程根伟.
  \newblock 1998年长江洪水的成因与减灾对策\allowbreak[M]//\allowbreak
  \highlightb{见:}许厚泽,赵其国.
  \newblock 长江流域洪涝灾害与科技对策.
  \newblock 北京:科学出版社,1999:32-36.

  \bibitem[Wang(2022)]{gbt7714.8.6.3:2}
  Wang Shanshan.
  \newblock Application of improved {SOM} neural network in intelligent auditing of hospital financial vouchers\allowbreak[C/OL]//\allowbreak
  \highlightb{In:} 2022 6th {Asian Conference} on {Artificial Intelligence Technology}({ACAIT}),2022:2.
  \newblock \url{https://ieeexplore.ieee.org/document/10137867}.
  \newblock DOI:\doi{10.1109/ACAIT56212.2022.10137867}.
\end{bibexample}


\subsection{刊名斜体}

\boolitem[false]{emphjournal}
控制刊名是否使用斜体。
该选项不影响中文文献。

\begin{bibexample}[title={\kvopt{emphjournal}{false}}]
  \bibitem[Caplan(1993)]{gbt7714.b.4:12}
  Caplan P.
  \newblock Cataloging internet resources\allowbreak[J].
  \newblock \highlighta{The Public-Access Computer Systems Review},1993,4\allowbreak (2):61-66.
\end{bibexample}

\begin{bibexample}[title={\kvopt{emphjournal}{true}}]
  \bibitem[Caplan(1993)]{gbt7714.b.4:12}
  Caplan P.
  \newblock Cataloging internet resources\allowbreak[J].
  \newblock \highlightb{\emph{The Public-Access Computer Systems Review}},1993,4\allowbreak (2):61-66.
\end{bibexample}



\subsection{刊名缩写}

\boolitem[false]{shortjournal}
控制刊名是否使用缩写。需要在 \field{shortjournal} 字段填写刊名缩写。

\begin{bibtexfile}
@article{gbt7714.b.4:12,
  journal       = {The Public-Access Computer Systems Review},
  shortjournal  = {Public-Access Comput. Syst. Rev.},
  ...
}
\end{bibtexfile}

\begin{bibexample}[title={\kvopt{shortjournal}{false}}]
  \bibitem[Caplan(1993)]{gbt7714.b.4:12}
  Caplan P.
  \newblock Cataloging internet resources\allowbreak[J].
  \newblock \highlighta{The Public-Access Computer Systems Review},1993,4\allowbreak (2):61-66.
\end{bibexample}

\begin{bibexample}[title={\kvopt{shortjournal}{true}}]
  \bibitem[Caplan(1993)]{gbt7714.8.1:4}
  Caplan P.
  \newblock Cataloging internet resources\allowbreak[J].
  \newblock \highlightb{Public-Access Comput Syst Rev},1993,4\allowbreak (2):61-66.
\end{bibexample}


\subsection{刊名移除缩写点}

\boolitem[true]{dotlessjournal}
控制刊名是否使用缩写。需要在 \field{shortjournal} 字段填写刊名缩写。

\begin{bibexample}[title={\kvopt{shortjournal}{true}, \kvopt{dotlessjournal}{true}}]
  \bibitem[Caplan(1993)]{gbt7714.b.4:12}
  Caplan P.
  \newblock Cataloging internet resources\allowbreak[J].
  \newblock \highlighta{Public-Access Comput Syst Rev},1993,4\allowbreak (2):61-66.
\end{bibexample}

\begin{bibexample}[title={\kvopt{shortjournal}{true}, \kvopt{dotlessjournal}{false}}]
  \bibitem[Caplan(1993)]{gbt7714.8.1:4}
  Caplan P.
  \newblock Cataloging internet resources\allowbreak[J].
  \newblock \highlightb{Public-Access Comput. Syst. Rev.},1993,4\allowbreak (2):61-66.
\end{bibexample}


% \subsection{\opt{linktitle} 控制刊名是否有超链接}


\subsection{无出版地、无出版者}

\boolitem[false]{unknownpublisher}
控制无出版地或无出版者时是否著录“出版地不详”、“S.l.”、“出版者不详”、“s.n.”等内容。

\begin{bibexample}[title={\kvopt{unknownpublisher}{false}}]
  \bibitem[王夫之(1865(清同治四年))]{gbt7714.8.2.2:2}
  王夫之.
  \newblock 宋论\allowbreak[M].
  \newblock 刻本.
  \newblock 金陵,1865(清同治四年).

  \bibitem[Praetzellis(2011)]{gbt7714.b.1:22}
  Praetzellis A.
  \newblock Death by theory:a tale of mystery and archaeological theory\allowbreak[M/OL].
  \newblock Rev. ed.
  \newblock Rowman \& Littlefield Publishing Group,Inc.,2011:13.
  \newblock \url{http://lib.myilibrary.com/Open.aspx?id=293666}.
\end{bibexample}

\begin{bibexample}[title={\kvopt{unknownpublisher}{true}}]
  \bibitem[王夫之(1865(清同治四年))]{gbt7714.8.2.2:2}
  王夫之.
  \newblock 宋论\allowbreak[M].
  \newblock 刻本.
  \newblock 金陵:[出版者不详],1865(清同治四年).

  \bibitem[Praetzellis(2011)]{gbt7714.b.1:22}
  Praetzellis A.
  \newblock Death by theory:a tale of mystery and archaeological theory\allowbreak[M/OL].
  \newblock Rev. ed.
  \newblock [S.l.]:Rowman \& Littlefield Publishing Group,Inc.,2011:13.
  \newblock \url{http://lib.myilibrary.com/Open.aspx?id=293666}.
\end{bibexample}


% \subsection{\opt{spacebeforepages} 控制页码前是否有空格}


\subsection{起止页码}

\boolitem[true]{pageranges}
控制页码著录起止区间(“15-18”)或起始页(“15”)。


\subsection{页码分隔符}

\valitem[-]{pagerangedelim}{分隔符}
控制起止页码之间的分隔符,默认为短横线“-”。

常见的选择还有 en dash “--”、一字线“—”、波浪线“~”。


% \subsection{\opt{urldate} 控制是否著录引用日期}


\subsection{纸质文献著录 URL}

\boolitem[true]{url}

如果设为 \opt{false},纸质文献(如图书、期刊文章等)不再著录 URL。
但是电子资源(如网页、数据机、预印本)不受影响。

\begin{bibexample}[title={\kvopt{url}{true}}]
  \bibitem[赵学功(2001)]{gbt7714.b.1:14}
  赵学功.
  \newblock 当代美国外交\allowbreak[M/OL].
  \newblock 北京:社会科学文献出版社,2001.
  \newblock \highlighta{\url{http://www.cadal.zju.edu.cn/book/trySinglePage/33023884/1}}.

  \bibitem[Jenkins et~al.(2012)Jenkins and Ruostekoski]{gbt7714.8.15.2:3}
  Jenkins S~D,Ruostekoski J.
  \newblock Controlled manipulation of light by cooperative response of atoms in an optical lattice\allowbreak[PP/OL].
  \newblock V2.
  \newblock arXiv\allowbreak (2012-03-18)\allowbreak[2020-06-24].
  \newblock \highlighta{\url{https://doi.org/10.48550/arXiv.1112.6136}}.
\end{bibexample}

\begin{bibexample}[title={\kvopt{url}{true}}]
  \bibitem[赵学功(2001)]{gbt7714.b.1:14}
  赵学功.
  \newblock 当代美国外交\allowbreak[M/OL].
  \newblock 北京:社会科学文献出版社,2001.

  \bibitem[Jenkins et~al.(2012)Jenkins and Ruostekoski]{gbt7714.8.15.2:3}
  Jenkins S~D,Ruostekoski J.
  \newblock Controlled manipulation of light by cooperative response of atoms in an optical lattice\allowbreak[PP/OL].
  \newblock V2.
  \newblock arXiv\allowbreak (2012-03-18)\allowbreak[2020-06-24].
  \newblock \highlighta{\url{https://doi.org/10.48550/arXiv.1112.6136}}.
\end{bibexample}


\subsection{DOI}

\boolitem[true]{doi}

控制是否著录 DOI。此选项还同时控制是否著录其他永久标识符(如 \href{https://cstr.cn/}{CSTR})。

\begin{bibexample}[title={\kvopt{doi}{true}}]
  \bibitem[肖玲\ 等(2024)肖玲,张雪和王永]{gbt7714.8.15.2:1}
  肖玲,张雪,王永.
  \newblock 数据要素的统计测算方法探究\allowbreak[PP/OL].
  \newblock PSSXiv\allowbreak (2024-07-02)\allowbreak[2024-09-30].
  \newblock \url{https://zsyyb.cn/abs/202408.01096}.
  \newblock \highlighta{CSTR:\cstr{32012.36.PSSXiv.202408.01096}}.

  \bibitem[Frese et~al.(2013)Frese,Katus,and Meder]{gbt7714.b.4:16}
  Frese K~S,Katus H~A,Meder B.
  \newblock Next-generation sequencing:from understanding biology to personalized medicine\allowbreak[J/OL].
  \newblock Biology,2013,2\allowbreak (1):378-398.
  \newblock \url{http://www.mdpi.com/2079-7737/2/1/378}.
  \newblock \highlighta{DOI:\doi{10.3390/biology2010378}}.
\end{bibexample}

\begin{bibexample}[title={\kvopt{doi}{false}}]
  \bibitem[肖玲\ 等(2024)肖玲,张雪和王永]{gbt7714.8.15.2:1}
  肖玲,张雪,王永.
  \newblock 数据要素的统计测算方法探究\allowbreak[PP/OL].
  \newblock PSSXiv\allowbreak (2024-07-02)\allowbreak[2024-09-30].
  \newblock \url{https://zsyyb.cn/abs/202408.01096}.

  \bibitem[Frese et~al.(2013)Frese,Katus,and Meder]{gbt7714.b.4:16}
  Frese K~S,Katus H~A,Meder B.
  \newblock Next-generation sequencing:from understanding biology to personalized medicine\allowbreak[J/OL].
  \newblock Biology,2013,2\allowbreak (1):378-398.
  \newblock \url{http://www.mdpi.com/2079-7737/2/1/378}.
\end{bibexample}


\subsection{预印本标识符}

\boolitem[false]{eprint}

控制是否著录预印本的标识符(\field{eprint} 字段)。

\begin{bibexample}[title={\kvopt{eprint}{false}}]
  \bibitem[肖玲\ 等(2024)肖玲,张雪和王永]{gbt7714.8.15.2:1}
  肖玲,张雪,王永.
  \newblock 数据要素的统计测算方法探究\allowbreak[PP/OL].
  \newblock PSSXiv\allowbreak (2024-07-02)\allowbreak[2024-09-30].
  \newblock \url{https://zsyyb.cn/abs/202408.01096}.
  \newblock CSTR:\cstr{32012.36.PSSXiv.202408.01096}.

  \bibitem[Jenkins et~al.(2012)Jenkins and Ruostekoski]{gbt7714.8.15.2:3}
  Jenkins S~D,Ruostekoski J.
  \newblock Controlled manipulation of light by cooperative response of atoms in an optical lattice\allowbreak[PP/OL].
  \newblock V2.
  \newblock arXiv\allowbreak (2012-03-18)\allowbreak[2020-06-24].
  \newblock \url{https://doi.org/10.48550/arXiv.1112.6136}.
\end{bibexample}

\begin{bibexample}[title={\kvopt{eprint}{true}}]
  \bibitem[肖玲\ 等(2024)肖玲,张雪和王永]{gbt7714.8.15.2:1}
  肖玲,张雪,王永.
  \newblock 数据要素的统计测算方法探究\allowbreak[PP/OL].
  \newblock PSSXiv\highlightb{:202408.01096}\allowbreak (2024-07-02)\allowbreak[2024-09-30].
  \newblock \url{https://zsyyb.cn/abs/202408.01096}.
  \newblock CSTR:\cstr{32012.36.PSSXiv.202408.01096}.

  \bibitem[Jenkins et~al.(2012)Jenkins and Ruostekoski]{gbt7714.8.15.2:3}
  Jenkins S~D,Ruostekoski J.
  \newblock Controlled manipulation of light by cooperative response of atoms in an optical lattice\allowbreak[PP/OL].
  \newblock V2.
  \newblock arXiv\highlightb{:1112.6136}\allowbreak (2012-03-18)\allowbreak[2020-06-24].
  \newblock \url{https://doi.org/10.48550/arXiv.1112.6136}.
\end{bibexample}


\subsection{著录补充信息}

\boolitem[false]{note}

控制是否著录补充信息(\field{note} 字段)。


\subsection{文献结尾句点}

\boolitem[true]{enddot}

控制每一条文献结尾是否用“.”号标识。







\end{optionlist}


\section{与其他宏包的兼容性}



\subsection{bibunits}

\subsection{chapterbib}


\bibliography{\jobname}

\end{document}
