%
% Copyright (c) 2016-2025 Zeping Lee
% Released under the LaTeX Project Public License v1.3c or later versions.
% Repository: https://github.com/zepinglee/gbt7714-bibtex-style
%

\documentclass[a4paper]{l3doc}
\usepackage{hypdoc}
\usepackage[UTF8]{ctex}
\usepackage{fontspec}
\usepackage{caption}
\usepackage{booktabs}
\usepackage{gbt7714}
\usepackage{hologo}
\usepackage{listings}
\makeatletter
\hypersetup{allcolors=blue}
\setmainfont{texgyretermes}[
  Extension      = .otf,
  UprightFont    = *-regular,
  BoldFont       = *-bold,
  ItalicFont     = *-italic,
  BoldItalicFont = *-bolditalic,
]%
\setsansfont{texgyreheros}[
  Extension      = .otf,
  UprightFont    = *-regular,
  BoldFont       = *-bold,
  ItalicFont     = *-italic,
  BoldItalicFont = *-bolditalic,
]%
\setmonofont{texgyrecursor}[
  Extension      = .otf,
  UprightFont    = *-regular,
  BoldFont       = *-bold,
  ItalicFont     = *-italic,
  BoldItalicFont = *-bolditalic,
  Scale          = MatchLowercase,
  Ligatures      = CommonOff,
]%
\citestyle{super}
\lstnewenvironment{latex}{%
  \lstset{
    basicstyle = \ttfamily\small,
    language = [LaTeX]TeX,
    gobble = 4,
    frame = single,
  }%
}{}
\lstnewenvironment{pseudocode}{%
  \lstset{
    basicstyle=\ttfamily\small,
    language=bash,
    gobble=2,
    frame=single,
  }%
}{}
\DeclareRobustCommand\file{\nolinkurl}
\DeclareRobustCommand\env{\texttt}
\DeclareRobustCommand\pkg{\textsf}
\DeclareRobustCommand\cls{\textsf}
\DeclareRobustCommand\opt{\texttt}
\setlength\partopsep{\z@}
\def\@listi{\leftmargin\leftmargini
            \parsep \z@
            \topsep 5\p@ \@plus2\p@ \@minus3\p@
        \itemsep\z@}
\let\@listI\@listi
\@listi
\makeatother
\EnableCrossrefs
\CodelineIndex
% \OnlyDescription


\begin{document}

\title{GB/T 7714 \hologo{BibTeX} style}
\author{Zeping Lee\thanks{zepinglee AT gmail.com}}
\date{\filedate\qquad\fileversion}
\maketitle



\begin{abstract}
The \pkg{gbt7714} package provides a \BibTeX{} implementation for the China's
national bibliography style standard GB/T 7714.
It consists of \file{.bst} files for numeric and author-date styles as well as a
\LaTeX{} package which provides the citation style defined in the standard.
It is compatible with \pkg{natbib} and supports language detection (Chinese
and English) for each biblilography entry.
\end{abstract}

\section{简介}

GB/T 7714—2015 《信息与文献\quad 参考文献著录规则》\cite{gbt7714-2015}
(以下简称“国标”)是中国的参考文献格式推荐标准。
国内的绝大部分学术期刊、学位论文都使用了基于该标准的格式。
本宏包是国标的 \BibTeX{}\cite{bibtex} 实现,具有以下特性:
\begin{itemize}
  \item 兼容 \pkg{natbib} 宏包\cite{natbib}。
  \item 支持“顺序编码制”和“著者-出版年制”两种风格。
  \item 自动识别语言并进行相应处理。
  \item 提供了简单的接口供用户修改样式。
  \item 同时提供了 2005 版的 \file{.bst} 文件。
\end{itemize}
本宏包的主页:\url{https://github.com/zepinglee/gbt7714-bibtex-style}。


\section{版本 v2.0 的重要修改}

从 v2.0 版本开始(2020-03-04),用户必须在文档中使用 \cs{biblilographystyle} 命令选择参考文献样式,
如 \opt{gbt7714-numerical} 或 \opt{gbt7714-author-year}。
在早期的版本中,选择文献样式的方法是将 \opt{numbers} 或 \opt{super} 等参数传递给
\pkg{gbt7714},
而不能使用 \cs{bibliographystyle}。
这跟标准的 LaTeX 接口不一致,所以将被弃用。


\section{使用方法}

以下是 \pkg{gbt7714} 宏包的一个简单示例。
\begin{latex}
  \documentclass[UTF8]{ctexart}
  \usepackage{gbt7714}
  \bibliographystyle{gbt7714-numerical}
  \begin{document}
    \cite{...}
    ...
    \bibliography{bibfile}
  \end{document}
\end{latex}

按照国标的规定,参考文献的标注体系分为“顺序编码制”和
“著者-出版年制”。
用户应在导言区调用宏包 \pkg{gbt7714},并且使用 \cs{bibliographystyle}
命令选择参考文献表的样式,比如:
\begin{latex}
  \bibliographystyle{gbt7714-numerical}  % 顺序编码制
\end{latex}
或者
\begin{latex}
  \bibliographystyle{gbt7714-author-year}  % 著者-出版年制
\end{latex}
此外还可以使用 2005 版的格式 \file{gbt7714-2005-numerical} 和
\file{gbt7714-2005-author-year}。

注意,版本 v2.0 更改了设置参考文献表样式的方法,
要求直接使用 \cs{bibliographystyle},不再使用宏包的参数,
而且更改了 \file{bst} 的文件名。

\begin{function}{\citestyle}
  \begin{syntax}
    |\citestyle|\Arg{citation style}
  \end{syntax}
可选:\opt{super}, \opt{numbers}, \opt{author-year}。
使用 \cs{bibliography} 选择参考文献表的样式时会自动设置对应的引用样式。
顺序编码制的引用标注默认使用角标式(\opt{super}),
如“张三\textsuperscript{[2]}提出”。
如果要使用正文模式,如“文献 [3] 中说明”,
可以使用 \cs{citestyle} 命令切换为数字式(\opt{numbers})。
\begin{latex}
  \citestyle{numbers}
\end{latex}
著者-出版年制通常不需要修改引用样式。
\end{function}

\DescribeOption{sort\&compress}
同一处引用多篇文献时,应当将各篇文献的 key 一同写在 \cs{cite} 命令中。
如遇连续编号,默认会自动转为起讫序号并用短横线连接
(见\pkg{natbib} 的 \opt{compress} 选项)。
如果要对引用的编号进行自动排序,需要在调用 \pkg{gbt7714} 时加
\opt{sort\&compress} 参数,
这些参数会传给 \pkg{natbib} 处理。
\begin{latex}
  \usepackage[sort&compress]{gbt7714}
\end{latex}
注意国标中要求 2 个或以上的连续编号用连接号,不同于 \pkg{natbib} 默认的 3 个或以上。
宏包中已经作了修改。

若需要标出引文的页码,可以标在 \cs{cite} 的可选参数中,如
|\cite[42]{knuth84}|。
更多的引用标注方法可以参考 \pkg{natbib} 宏包的使用说明\cite{natbib}。

\DescribeOption{locator-inside-brackets}
国标要求在括号外以角标的形式著录引文页码。
如果要将页码置于括号内,可以在调用宏包时设置 \opt{locator-inside-brackets=true}。
\begin{latex}
  \usepackage[locator-inside-brackets=true]{gbt7714}
\end{latex}

使用时需要注意以下几点:
\begin{itemize}
  \item \file{.bib} 数据库应使用 UTF-8 编码。
  \item 使用著者-出版年制参考文献表时,中文的文献\emph{必须}
        在 key 域填写作者姓名的拼音,才能按照拼音排序,
        详见第~\ref{sec:sort}~节。
\end{itemize}


\section{文献类型}

国标中规定了 16 种参考文献类型,
表~\ref{tab:entry-types} 列举了 \file{bib} 数据库中对应的文献类型。
这些尽可能兼容 \BibTeX{} 和 \pkg{biblatex} 的标准类型,但是新增了若干文献类型(带 * 号)。

\begin{table}[htb]
  \centering\small
  \caption{全部文献类型}
  \label{tab:entry-types}
  \begin{tabular}{lcl}
    \toprule
    文献类型         & 标识代码 & Entry Type                  \\
    \midrule
    普通图书         & M        & book                        \\
    图书的析出文献   & M        & incollection                \\
    会议录           & C        & proceedings                 \\
    会议录的析出文献 & C        & inproceedings 或 conference \\
    汇编             & G        & collection*                 \\
    报纸             & N        & newspaper*                  \\
    期刊的析出文献   & J        & article                     \\
    学位论文         & D        & mastersthesis 或 phdthesis  \\
    报告             & R        & techreport                  \\
    标准             & S        & standard*                   \\
    专利             & P        & patent*                     \\
    数据库           & DB       & database*                   \\
    计算机程序       & CP       & software*                   \\
    电子公告         & EB       & online*                     \\
    档案             & A        & archive*                    \\
    舆图             & CM       & map*                        \\
    数据集           & DS       & dataset*                    \\
    其他             & Z        & misc                        \\
    \bottomrule
  \end{tabular}
\end{table}


\section{著录项目}

由于国标中规定的著录项目多于 \BibTeX{} 的标准域,
必须新增一些著录项目(带 * 号),
这些新增的类型在设计时参考了 BibLaTeX,如 date 和 urldate。
本宏包支持的全部域如下:
\begin{description}
  \item[author]       主要责任者
  \item[title]        题名
  \item[mark*]        文献类型标识
  \item[medium*]      载体类型标识
  \item[translator*]  译者
  \item[editor]       编辑
  \item[organization] 组织(用于会议)
  \item[booktitle]    图书题名
  \item[series]       系列
  \item[journal]      期刊题名
  \item[edition]      版本
  \item[address]      出版地
  \item[publisher]    出版者
  \item[school]       学校(用于 \texttt{@phdthesis})
  \item[institution]  机构(用于 \texttt{@techreport})
  \item[year]         出版年
  \item[volume]       卷
  \item[number]       期(或者专利号)
  \item[pages]        引文页码
  \item[date*]        更新或修改日期
  \item[urldate*]     引用日期
  \item[url]          获取和访问路径
  \item[doi]          数字对象唯一标识符
  \item[langid*]      语言
  \item[key]          拼音(用于排序)
\end{description}
不支持的 \BibTeX{} 标准著录项目有 annote, chapter, crossref, month, type。

本宏包默认情况下可以自动识别文献语言,并自动处理文献类型和载体类型标识,
但是在少数情况下需要用户手动指定,如:
\begin{latex}
  @misc{citekey,
    langid = {japanese},
    mark   = {Z},
    medium = {DK},
    ...
  }
\end{latex}
可选的语言有 english, chinese, japanese, russian。


\section{文献列表的排序}
\label{sec:sort}

国标规定参考文献表采用著者-出版年制组织时,各篇文献首先按文种集中,
然后按著者字顺和出版年排列;
中文文献可以按著者汉语拼音字顺排列,也可以按著者的笔画笔顺排列。
然而由于 \BibTeX{} 功能的局限性,无法自动获取著者姓名的拼音或笔画笔顺,
所以\emph{必须}在 bib 数据库中的 key 域手动录入著者姓名的拼音用于排序,如:
\begin{latex}
  @book{capital,
    author = {马克思 and 恩格斯},
    key    = {ma3 ke4 si1 & en1 ge2 si1},
    ...
  }
\end{latex}

对于著者-出版年的样式,如果中文文献较多时更推荐使用 \pkg{biblatex} 宏包,
其后端 \file{biber} 可以自动处理中文按照拼音排序,无须手动填写拼音。


\section{自定义样式}

\BibTeX{} 对自定义样式的支持比较有限,
所以用户只能通过修改 \file{bst} 文件来修改文献列表的格式。
本宏包提供了一些接口供用户更方便地修改。

在 \file{bst} 文件开始处的 |load.config| 函数中,
有一组配置参数用来控制样式,表~\ref{tab:config} 列出了每一项的默认值和功能。
若变量被设为 |#1| 则表示该项被启用,设为 |#0| 则不启用。
默认的值是严格遵循国标的配置。

\begin{table}[htb]
\centering\small
\caption{参考文献表样式的配置参数}
\label{tab:config}
\begin{tabular}{lcl}
  \toprule
  参数值                         & 默认值 & 功能                           \\
  \midrule
  uppercase.name                 & |#1|   & 将著者姓名转为大写             \\
  max.num.authors                & |#3|   & 输出著者的最多数量             \\
  year.after.author              & |#0|   & 年份置于著者之后               \\
  period.after.author            & |#0|   & 著者和年份之间使用句点连接     \\
  italic.book.title              & |#0|   & 西文书籍名使用斜体             \\
  sentence.case.title            & |#1|   & 将西文的题名转为 sentence case \\
  link.title                     & |#0|   & 在题名上添加 url 的超链接      \\
  title.in.journal               & |#1|   & 期刊是否显示标题               \\
  show.patent.country            & |#0|   & 专利题名是否含国别             \\
  space.before.mark              & |#0|   & 文献类型标识前是否有空格       \\
  show.mark                      & |#1|   & 显示文献类型标识               \\
  show.medium.type               & |#1|   & 显示载体类型标识               \\
  component.part.label           & |"slash"| & 表示析出文献的符号,可选:|"in"|, |"none"| \\
  italic.journal                 & |#0|   & 西文期刊名使用斜体             \\
  link.journal                   & |#0|   & 在期刊题名上添加 url 的超链接      \\
  show.missing.address.publisher & |#0|   & 出版项缺失时显示“出版者不详”   \\
  space.before.pages             & |#1|   & 页码与前面的冒号之间有空格     \\
  only.start.page                & |#0|   & 只显示起始页码                 \\
  page.range.delimiter           & |"-"|  & 起止页码中的连接号             \\
  show.urldate                   & |#1|   & 显示引用日期 urldate           \\
  show.url                       & |#1|   & 显示 url                       \\
  show.doi                       & |#1|   & 显示 DOI                       \\
  show.preprint                  & |#1|   & 显示预印本信息                 \\
  show.note                      & |#0|   & 显示 note 域的信息             \\
  end.with.period                & |#1|   & 结尾加句点                     \\
  lowercase.word.after.colon     & |#1|   & 将冒号后的单词变成小写 \\
  \bottomrule
\end{tabular}
\end{table}

若用户需要定制更多内容,可以学习 \file{bst} 文件的语法并修改
\cite{btxhak,ttb,tlc2},或者联系作者。


\section{相关工作}

TeX 社区也有其他关于 GB/T 7714 系列参考文献标准的工作。
2005 年吴凯\cite{wk2006}发布了基于 GB/T 7714—2005 的 \BibTeX{} 样式,
支持顺序编码制和著者出版年制两种风格。
李志奇\cite{lqz2013}发布了严格遵循 GB/T 7714—2005 的 BibLaTeX 的样式。
胡海星\cite{hhx2013}提供了另一个 \BibTeX{} 实现,
还给每行 bst 代码写了 java 语言注释。
沈周\cite{sz2016}基于 \pkg{biblatex-caspervector}\cite{vector2012} 进行修改,
以符合国标的格式。
胡振震发布了符合 GB/T 7714—2015 标准的 BibLaTeX 参考文献样式\cite{hzz2016},
并进行了比较完善的持续维护。


\begin{thebibliography}{12}
\providecommand{\natexlab}[1]{#1}
\providecommand{\url}[1]{#1}
\expandafter\ifx\csname urlstyle\endcsname\relax\relax\else
  \urlstyle{same}\fi
\providecommand{\href}[2]{\url{#2}}
\providecommand{\doi}[1]{\href{https://doi.org/#1}{#1}}

\bibitem[{中国国家标准化委员会}(2015)]{gbt7714-2015}
{中国国家标准化委员会}.
\newblock 信息与文献\quad 参考文献著录规则: GB/T
  7714—2015[S].
\newblock 北京: 中国标准出版社, 2015.

\bibitem[Patashnik(1988{\natexlab{a}})]{bibtex}
PATASHNIK~O.
\newblock {\BibTeX}ing[M/OL].
\newblock 1988{\natexlab{a}}.
\newblock \url{http://mirrors.ctan.org/biblio/bibtex/base/btxdoc.pdf}.

\bibitem[Daly(1999)]{natbib}
DALY~P~W.
\newblock Natural sciences citations and references[M/OL].
\newblock 1999.
\newblock \url{http://mirrors.ctan.org/macros/latex/contrib/natbib/natbib.pdf}.

\bibitem[Patashnik(1988{\natexlab{b}})]{btxhak}
PATASHNIK~O.
\newblock Designing {\BibTeX} styles[M/OL].
\newblock 1988{\natexlab{b}}.
\newblock \url{http://mirrors.ctan.org/biblio/bibtex/base/btxhak.pdf}.

\bibitem[Markey(2003)]{ttb}
MARKEY~N.
\newblock Tame the beast[M/OL].
\newblock 2003.
\newblock \url{http://mirrors.ctan.org/info/bibtex/tamethebeast/ttb_en.pdf}.

\bibitem[Mittelbach et~al.(2004)Mittelbach, Goossens, Braams, Carlisle, and
  Rowley]{tlc2}
MITTELBACH~F, GOOSSENS~M, BRAAMS~J, et~al.
\newblock The {\LaTeX} companion[M].
\newblock 2nd ed.
\newblock Reading, MA, USA: Addison-Wesley, 2004.

\bibitem[吴凯(2006)]{wk2006}
吴凯.
\newblock 发布GBT7714-2005.bst version1 Beta版[EB/OL].
\newblock 2006.
\newblock CTeX 论坛(已关闭).

\bibitem[李志奇(2013)]{lqz2013}
李志奇.
\newblock
  基于biblatex的符合GBT7714—2005的中文文献生成工具[EB/OL].
\newblock 2013.
\newblock CTeX 论坛(已关闭).

\bibitem[胡海星(2013)]{hhx2013}
胡海星.
\newblock A GB/T 7714—2005 national standard compliant BibTeX
  style[EB/OL].
\newblock 2013.
\newblock \url{https://github.com/Haixing-Hu/GBT7714-2005-BibTeX-Style}.

\bibitem[沈周(2016)]{sz2016}
沈周.
\newblock 基于caspervector改写的符合GB/T
  7714—2005标准的参考文献格式[EB/OL].
\newblock 2016.
\newblock \url{https://github.com/szsdk/biblatex-gbt77142005}.

\bibitem[Vector(2012)]{vector2012}
VECTOR~C~T.
\newblock biblatex 参考文献和引用样式: caspervector[M/OL].
\newblock 2012.
\newblock
  \url{http://mirrors.ctan.org/macros/latex/contrib/biblatex-contrib/biblatex-caspervector/doc/caspervector.pdf}.

\bibitem[胡振震(2016)]{hzz2016}
胡振震.
\newblock 符合 GB/T 7714—2015 标准的 biblatex
  参考文献样式[M/OL].
\newblock 2016.
\newblock
  \url{http://mirrors.ctan.org/macros/latex/contrib/biblatex-contrib/biblatex-gb7714-2015/biblatex-gb7714-2015.pdf}.

\end{thebibliography}




\clearpage
\appendix
\StopEventually{}

\end{document}
